\section{Présentation}

  \begin{paragraphe}
  	Le projet que nous réalisons est un site de revues participatives d’albums
    musicaux.\par
    Il met à disposition une large base de données comportant de nombreux artistes
    pour lesquels on retrouve une brève description ainsi que leur discographie.
    Pour chaque album on donne les chansons présentes et quelques informations
    supplémentaires (durée, date de parution, artistes, label, genre, etc). \par
    Il est également possible de s’identifier afin de pouvoir rédiger des
    commentaires et des évaluations pour chaque album présent sur le site.
  \end{paragraphe}

  \begin{paragraphe}
    Ainsi, le site final est un recueil d’albums pour lesquels on retrouve
    quelques informations et de nombreux avis d’utilisateurs.
  \end{paragraphe}

  \subsection{Systèmes comparables}

  \begin{paragraphe}
    Il existe le site Pitchfork (https://pitchfork.com/), site anglo-saxon
    spécialisé dans le revue d’albums de tout genre musical. Cependant ce site
    offre des revues rédigées par des auteurs spécialisés appartenant à la
    société Pitchfork et n’offrent pas plus de données sur l’album mis à part
    la revue.\par
    Le site Rotten Tomatoes (https://www.rottentomatoes.com/), site de revue de
    films, se rapproche plus de notre projet car il permet aux utilisateurs de
    donner un avis et fournit donc la note moyenne accordée par les
    utilisateurs à une œuvre, et donne également des données sur le film
    (durée, date de parution).
  \end{paragraphe}

  \subsection{Profils et rôles utilisateurs}

  \begin{paragraphe}
    Nous utilisons deux profils d’acteurs à travers ce projet, le profil
    administrateur et le profil utilisateur.\par
    En tant que créateurs du site, nous sommes les administrateurs de celui-ci
    et avons donc la charge de mettre le site en place, de le rendre
    fonctionnel et de remplir la base de données (saisir les artistes, les
    albums, les pistes ainsi que les informations pour chaque).\par
    Une fois le site fonctionnel et rempli, les utilisateurs (à qui on demande
    une identification, authentification) peuvent consulter les artistes et les
    albums disponibles et peuvent proposer des évaluations pour les albums de
    leur choix.
  \end{paragraphe}

  \subsection{Contraintes}

  \begin{paragraphe}
    Les difficultés de ce projet sont le temps disponible ainsi que le volume
    de la base de données à fournir avant de pouvoir déclarer le site comme
    fonctionnel. De plus, il faut réfléchir à beaucoup d’éléments pour ne pas
    que l’ajout des évaluations sature ou déstabilise la mise en page déjà
    effectuée.
  \end{paragraphe}

\section{Fonctionnalités}

  \subsection{Terminologie}

    \begin{paragraphe}
      Tout d'abord on va spécifier quelques termes utilisés dans ce cachier des
      charges.
      \begin{itemize}
        \item \textbf{Artiste :} Définit le compositeur d'un album, d'un
          morceau de musique ou ayant participer à un album ou un morceau de
          musique.
        \item \textbf{Album :} Est un receuil de morceaux de musique d'un ou
          plusieurs artistes.
        \item \textbf{Morceau de musique :} Est une musique faisant partie d'un
          album d'une artiste.
        \item \textbf{Single :} Est une musique ne faisant pas partie d'un album
          mais directement relié à un artiste.
        \item \textbf{Métadonnées :} Ensemble d'éléments caractérisant un
          artiste, un album ou un morceau de musique.
        \item \textbf{Label :} Communément appelée \emph{maison de disques}
          elle produit, distribue les enregistrements d'un artiste.
      \end{itemize}
    \end{paragraphe}

  \subsection{Parties de l'application}

    \begin{paragraphe}
      Il existe deux parties dans l'application. Une partie publique et une partie
      privée.
    \end{paragraphe}

    \begin{paragraphe}
      \textbf{La partie public} est la vitrine de l'application. C'est ici que l'on
      va retrouver les différentes fonctionnalités citées plus bas. On peut
      ainsi consulter des artistes, les albums d'un artiste et ou des chansons.
      De plus on peut effectuer des recherches plus ou moins précises.
      Ainsi que ajouter des commentaires avec ou sans notation.

      \textbf{La partie privée} est réservée au modérateur de l'application web.
      Elle permet entre autre de gérer toute la base de données des artistes,
      album, single et métadonnées de ces derniers. On peut ajouter, modifier
      ou supprimer une entrée dans la base de données.
    \end{paragraphe}

  \subsection{Description des fonctionnalités}

    \begin{paragraphe}
      	Le site offre la possibilité de créer un compte, de consulter des pages
        (artistes, albums), de rédiger des évaluations et des revues.
    \end{paragraphe}

    \begin{paragraphe}
      Voici les différentes fonctionnalités disponible dans l'application web.
    \end{paragraphe}

    \begin{paragraphe}
      \begin{itemize}
        \item Effectuer une recherche d'un artiste, d'un album ou d'un single
          par son nom, sa date, son genre.
        \item Consulter la description d'un artiste. On retrouve son nom, sa
          date de naissance, une brève description et une discographie.
        \item Consulter la description d'un album. On retrouve son nom,
          l'artiste compositeur et une liste d'artiste ayant participée. De
          même que son année de parution et la liste des morceaux de musique
          dont il le compose.
        \item Consulter les métadonnées d'un morceau de musique. On retrouve
          le nom, l'album rattaché ainsi que son autheur. L'année de composition
          et son genre.
      \end{itemize}
    \end{paragraphe}

    \begin{paragraphe}
      \begin{itemize}
        \item Créer un compte utilisateur.
        \item Se connecter à un compte enregistré.
        \item Modifier son mot de passe.
        \item Supprimer son compte utilisateur.
      \end{itemize}
    \end{paragraphe}

    \begin{paragraphe}
      \begin{itemize}
        \item Déposer une critique sur une page d'un artiste, d'un album ou
          d'une musique. Avec une note ou non.
        \item Modifier une critique dont on est l'autheur.
      \end{itemize}
    \end{paragraphe}

    \begin{paragraphe}
      Pour la partie privée voici les fonctionnalités de gestion de la base de
      données.
    \end{paragraphe}

    \begin{paragraphe}
      \begin{itemize}
        \item Ajouter un artiste avec toutes ces métadonnées.
        \item Modifier les métadonnées d'un artiste.
        \item Supprimer un artistes, supprime aussi ces albums et les titres de
          musique
        \item Ajouter un album à un artiste.
        \item Modifier un album d'un artiste.
        \item Supprime un album d'un artiste, ainsi que les morceaux de musique
          associé.
        \item Ajouter un morceau de musique à un album d'un artiste avec toutes
          ces métadonnées.
        \item Ajouter un single à un artiste avec toutes ces métadonnées.
        \item Modifier les métadonnées d'un morceau de musique à un album
          d'un artiste.
        \item Modifier les métadonnées d'un single d'un artiste.
        \item Supprimer un morceau de musique à un album d'un artiste.
        \item Supprimer un single à un artiste.
      \end{itemize}
    \end{paragraphe}

\section{Scénarios}

  \begin{paragraphe}
    Voici différents scénarios possibles d'un utilisateur interagissant avec
    l'application web.
  \end{paragraphe}

  \begin{paragraphe}
    \textbf{Scénario 1 :}
    Un utilisateur cherche le nom d'une musique pour obtenir des informations
    sur ce titre, comme l'album auquel il appartient, des informations sur les
    auteurs de ce morceau, la date de parution, etc Il regarde également l'avis
    des internautes en commentaire pour savoir ce qu'ils en ont pensé.
  \end{paragraphe}

  \begin{paragraphe}
    \textbf{Scénario 2 :}
    Pour l'anniversaire de sa fille Sophie, Teresa veut lui offrir un album
    musique. Teresa n'est cependant pas très familier avec les artistes
    d'aujourd'hui. Elle va donc consulter l'application en cherchant les
    artistes les plus populaires du moment. Elle trouve ainsi l'artiste
    M. Pokora qui a sorti son dernier album, lis la description et les
    commentaires très positive de l'album. Elle est donc convaincu et ira
    acheter l'album pour sa fille.
  \end{paragraphe}

  \begin{paragraphe}
    \textbf{Scénario 3 :}
    Ludovic est un fan de Ed sheeran et souhaite mettre un commentaire sur le
    site crée. \par
    Il met un commentaire sur la chanson "Casltle on the Hill" qui fait partie
    du 3ème single 2017 de l'artiste.
  \end{paragraphe}

  \begin{paragraphe}
    \textbf{Scénario 4 :}
    Un utilisateur ayant un compte et s'étant connecté, cherche un album qu'il
    a écouté et note chacun des morceaux en rédigeant ou non un commentaire.
  \end{paragraphe}

  \begin{paragraphe}
    \textbf{Scénario 5 :}
    Une personne se connecte sur la page d’accueil, effectue une recherche pour
    trouver la page de son groupe favori. Sur cette page il découvre la liste
    des albums parus par ce groupe, et décide de rédiger une évaluation de cette
    œuvre. Il s’identifie, rédige un avis et quitte la page. Sa session se
    ferme ensuite par le serveur après avoir remarqué une longue absence.
  \end{paragraphe}

  \begin{paragraphe}
    \textbf{Scénario 6 :}
    Une personne veut rédiger un avis sur un album, trouve la page
    correspondant mais ne possède pas de compte. Il décide donc de se créer un
    compte afin de pouvoir rédiger son évaluation.
  \end{paragraphe}
