
\section{Introduction}

	\begin{paragraphe}
		Ce document a pour but de décrire les aspects visuels et techniques de notre site.\\
		Ainsi, il présente les différentes interactions entre les différentes pages que nous avons conceptualisées.
		Il montre également sous quelles dispositions peuvent être retrouvé les différents éléments de notre site web et les moyens d'y accéder.\\
		Lors de la phase de développement du site, nous sommes revenus sur certains choix que nous avions faits lors de la conception.
		Nous nous attacherons donc ici à faire apparaître les différentes évolutions en prenant soin d'expliquer les raisons de
		ces changements et la réflexion que nous avons eue.
	\end{paragraphe}

\section{Charte graphique}

	\begin{paragraphe}
	    Nous avions d'abord opté pour une charte avec un fond beige entre deux bandeaux bleu azur, mais pour des raisons
	    d'esthétisme, de clarté et de lisibilité nous avons décidé de changer ces couleurs.\\

		Notre site a maintenant pour couleurs principales:
            \begin{itemize}
            \item une couleur d'écriture blanche.
            \item une couleur orangée pour les boutons, les mots-clés et le bas de page.
            \item une couleur gris sombre pour le fond et l'en tête de page.
            \end{itemize}
    \end{paragraphe}
    \image{0.50}{charte_graphique_couleurs}{Charte graphique - Couleurs}

    \begin{paragraphe}
		Le site comprend également deux bandeaux, un en haut et un en bas de notre chaque page.\\
		Le bandeau du haut comprend un bouton <<accueil>>,
		un bouton <<connexion>>, un bouton <<inscription>> si l'on n'est pas connecté et un bouton <<À propos>> contenant des metas-informations sur le site.
        \image{0.50}{bandeau_haut}{Charte graphique - Bandeau du haut}

        Le bandeau du bas rappelle le nom du site << Critique Musicale>> ainsi que la date de dernière maintenance du site.
    \end{paragraphe}
    \image{0.60}{bandeau_bas}{Charte graphique - Bandeau du bas}

\newpage

	\begin{paragraphe}
		Une barre de recherche est également présente en dessous du bandeau du haut sur toutes les pages du site, hormis celles de connexion, de gestion de compte et d'à-propos.
		Elle permet de saisir des recherches d'artiste, d'album, de groupe ou de morceau de musique.
	\end{paragraphe}
    \image{0.50}{barre_recherche}{Charte graphique - Barre de recherche}

	\begin{paragraphe}
		Le style et la taille de la police des caractères normaux pour toutes les pages sont par défaut respectivement \og hurme\_no2-webfont >> et << sans-sérif >>.\\
		Les titres sont en gras de couleur orangée. Ils ont pour style le << hurme\_no2-webfont >> en << sans-sérif >> avec une taille de police de 15px.
    \end{paragraphe}
    \image{0.43}{exemple_ecriture}{Charte graphique - Exemple du style d'écriture sur une page}


    \begin{paragraphe}
       Pour des affichages sous forme de tableau, nous avons choisi un design sobre, favorisant la lisibilité et la recherche rapide en alternant les couleurs des lignes
       et séparant distinctement les différentes lignes et colonnes du tableau.
    \end{paragraphe}
   \image{0.60}{tableau}{Charte graphique - Exemple d'un tableau}

\newpage

\section{Présentation}
	\subsection{Disposition 1 (layout)}

		\begin{paragraphe}
			Ceci est la première disposition dont toutes les pages vont hériter dans le cas où l'utilisateur n'est pas connecté.
		\end{paragraphe}

        \begin{center}
            \begin{tabular}{l c | c l}
                \textbf{Si} clic sur le bouton << accueil >> & & & \textbf{Si} clic sur le bouton << connexion >> \\
                \textbf{Alors} redirection vers \underline{pageAccueil.php}. & & & \textbf{Alors} redirection vers \underline{pageConnexion.php}.
            \end{tabular}
        \end{center}
        
        \begin{center}
            \begin{tabular}{l}
                \textbf{Si} clic sur bouton << inscription >> \\
                \textbf{Alors} redirection vers \underline{pageInscription.php}.
            \end{tabular}
        \end{center}

		\image{0.35}{page_type_1}{Mockup - Layout 1}

        \begin{paragraphe}
            Le layout final obtenu est très proche de ce que nous avions conceptualisé. Nous avons ajouté un bouton << À propos >>,
            déplacé le bouton << accueil >> sur la droite et affiché à sa place le titre du site qui dirige vers l'accueil lors de son activation.\\
            Nous avons également introduit la barre de recherche sur chaque page.
        \end{paragraphe}

        \image{0.30}{page_type_1_final}{Disposition 1 - Layout final 1}

    \newpage

	\subsection{Disposition 2 (layout)}

		\begin{paragraphe}
			Ceci est la seconde disposition dont toutes les pages vont hériter dans le cas où un utilisateur normal est connecté.
		\end{paragraphe}

        \begin{center}
            \begin{tabular}{l c | c l}
                \textbf{Si} clic sur le bouton << accueil >> & & & \textbf{Si} clic sur le bouton << compte >> \\
                \textbf{Alors} redirection vers \underline{pageAccueil.php}. & & & \textbf{Alors} redirection vers \underline{pageCompte.php}.
            \end{tabular}
        \end{center}
        
        \begin{center}
            \begin{tabular}{l}
                \textbf{Si} clic sur bouton << déconnexion >> \\
                \textbf{Alors} on déconnecte l'utilisateur de sa session courante.
            \end{tabular}
        \end{center}

		\image{0.35}{page_type_2}{Mockup - Layout 2}

        \begin{paragraphe}
            Le layout final obtenu est également très proche de ce que nous avions conceptualisé. Nous avons de nouveau ajouté un bouton << À propos >>,
            déplacé le bouton << accueil >> sur la droite et affiché à sa place le titre du site qui dirige vers l'accueil lors de son activation.\\
            Nous avons également introduit la barre de recherche sur chaque page. On remarque que le bouton << déconnexion >> est bien présent.
        \end{paragraphe}

        \image{0.30}{page_type_2_final}{Disposition 2 - Layout final 2}

	\newpage

	\subsection{Disposition 3 (layout)}

		\begin{paragraphe}
            Ceci est la troisième disposition dont toutes les pages vont hériter dans le cas où un utilisateur administrateur est connecté.
		\end{paragraphe}

        \begin{center}
            \begin{tabular}{l c | c l}
                \textbf{Si} clic sur le bouton << accueil >> & & & \textbf{Si} clic sur le bouton << administration >> \\
                \textbf{Alors} redirection vers \underline{pageAccueil.php}. & & & \textbf{Alors} redirection vers \underline{pageAdminAccueil.php}. \\ \\
                \textbf{Si} clic sur le bouton << compte >> & & & \textbf{Si} clic sur bouton << déconnexion >> \\
                \textbf{Alors} redirection vers \underline{pageCompte.php}. & & & \textbf{Alors} déconnexion de l'utilisateur de la session.
            \end{tabular}
        \end{center}

		\image{0.35}{page_type_3}{Mockup - Layout 3}

        \begin{paragraphe}
            Le layout final obtenu est également très proche de ce que nous avions conceptualisé. Nous avons de nouveau ajouté un bouton << À propos >>,
            déplacé le bouton << accueil >> sur la droite et affiché à sa place le titre du site qui dirige vers l'accueil lors de son activation.\\
            Nous avons également introduit la barre de recherche sur chaque page. On remarque que le bouton << administration >> est bien présent.
        \end{paragraphe}

        \image{0.30}{page_type_3_final}{Disposition 3 - final 3}

    \newpage

\section{Présentation cahier des charges}

\newpage

\section{Définition des pages}

	    	\begin{paragraphe}
	    	    Nous avions initialement prévu les pages suivantes pour composer notre site:\\
        		\begin{itemize}
        			\item \underline{pageAccueil.php} : Page d'accueil de notre site web.
        			\item \underline{pageRésultatRecherche.php} : Page des résultats des recherches entrée dans la barre de recherche par l'utilisateur.
        			\item \underline{pageArtisteGroupe.php} : Page des informations d'un artiste ou d'un groupe, ainsi que les albums et/ou singles du groupe ou de l'artiste.
        			\item \underline{pageAlbum.php} : Page d'affichage des données d'un morceau de musique.
        		\end{itemize}
                \vspace{1em}
                \begin{itemize}
        			\item \underline{pageConnexion.php} : Permet de se connecter au compte de l'utilisateur.
        			\item \underline{pageInscription.php} : Permet de créer un compte utilisateur.
        			\item \underline{pageCompte.php} : Permet d'afficher les informations de l'utilisateur et offre à l'utilisateur la possibilité de modifier son mot de passe.
        		\end{itemize}
                \vspace{1em}
        		\begin{itemize}
        			\item \underline{adminAccueil.php} : Permet de rediriger l'administrateur sur les différents panneaux de gestion de la base de données.
        			\item \underline{adminGestionArtiste.php} : Permet de gérer les différents artistes, en ajoutant ou en modifiant leurs données. Ainsi que leur attribuer des récompenses.
        			\item \underline{adminGestionGroupe.php} : Permet de gérer les différents groupes, en ajoutant ou en modifiant leurs données. Ainsi que de relier un ou plusieurs artistes à un groupe.
        			\item \underline{adminGestionAlbum.php} : Permet de gérer les différents albums, en ajoutant ou en modifiant leurs données. Ainsi que de relier un ou plusieurs artistes à un album.
        			\item \underline{adminGestionMusique.php} : Permet de gérer les différents morceaux de musique, en ajoutant ou en modifiant leurs données. Ainsi que de relier un ou plusieurs artistes à une musique, relier un ou plusieurs albums à une musique et de relier un ou plusieurs genres à une musique.
        			\item \underline{adminFormGenre.php} : Permet d'ajouter ou de modifier un genre à un morceau de musique ou un single.
        			\item \underline{adminFormRecompense.php} : Permet d'ajouter une récompense à un artiste.
        			\item \underline{adminGestionUtilisateur.php} : Permet à un administrateur préalablement identifié d'ajouter un utilisateur et/ou de changer le statut d'un utilisateur (administrateur ou non).
        		\end{itemize}
        	\end{paragraphe}

        	\begin{paragraphe}
                Cependant au fur et à mesure du développement nous nous sommes aperçus que par souci de modularité, de lisibilité, de factorisation et de simplicité,
                certains fichiers devaient être divisés en plusieurs quand d'autres pouvaient être fusionnés.
                De même certains on était ajoutés pour répondre aux fonctionnalités que nous n'avions pas conceptualisées.
            \end{paragraphe}

            \begin{paragraphe}
                Voici donc, page suivante, une liste exhaustive, tirée de notre documentation phpDoc générée à l'aide de Doxygen, présentant l'ensemble des fichiers .php avec un court descriptif de leur
                fonction rangé dans une arborescence.
            \end{paragraphe}

\newpage

            \image{0.43}{liste_fichier}{Définition des pages - Liste finale des fichiers}

    \clearpage

    \subsection{Pages publiques}

		\subsubsection{Page accueil}

            \begin{paragraphe}
                Nous avions originellement imaginé une page d'accueil contenant une barre de recherche, une liste des nouveaux artistes et des sorties récentes de musiques (voir Mockup de la page d'accueil).\\
                Nous avons cependant opté pour une conception plus sobre et simple en affichant simplement une grande barre de recherche (voir page d'accueil finale).
            \end{paragraphe}

			\begin{paragraphe}
                La barre de recherche a une place centrale sur la page d'accueil. Elle permet de saisir des recherches d’artiste, d’album, de groupe ou de morceau de musique.
				En entrant une recherche, les résultats seront affichés directement et différents liens seront disponibles.
                \image{0.50}{accueil_barre_recherche}{Page accueil - Barre recherche}
			\end{paragraphe}

			\begin{paragraphe}
				\textbf{Service PHP :}
            \end{paragraphe}

            \begin{paragraphe}
                Redirection vers la page recherche.php qui traitera la requête soumise dans le formulaire de la barre de recherche.
			\end{paragraphe}

			\begin{paragraphe}
				\textbf{Conception IHM :}
            \end{paragraphe}

            \image{0.30}{page_accueil}{Mockup - Page d'accueil}

            \begin{paragraphe}
                \textbf{Si} clic sur bouton << rechercher >> \\
                \textbf{Alors} redirige vers la page des résultats recherche en passant en paramètre la chaîne présente dans la barre de recherche
            \end{paragraphe}

             \image{0.30}{page_accueil_finale}{Pages publiques - Page d'accueil finale}

        \clearpage

		\subsubsection{Page résultat recherche}

			\begin{paragraphe}
				Cette page fait des requêtes sur la base de données, sur les entités << Artiste >>, << Musique >>, << Groupe>> et << Album >>.
				 en passant en paramètre la chaine de caractère entrée dans la barre de recherche.\\
                En cliquant sur le résultat souhaité, l'utilisateur est redirigé vers la page de l'artiste, du groupe, de l'album ou de la musique
                 recherché où il y trouve des informations détaillées caractérisant l'entité sélectionnée.
            \end{paragraphe}

            \begin{paragraphe}
                \textbf{Service PHP :}
            \end{paragraphe}

            \begin{paragraphe}
                Les services utilisés pour la recherche (Album, artiste, groupe ou musique):
                 \begin{itemize}
                     \item \emph{recuperer\_album(recherche)} affiche tous les albums correspondants à la chaîne recherchée.
                     \item \emph{recuperer\_artiste(recherche)} affiche tous les artistes correspondants à la chaîne recherchée.
                     \item \emph{recuperer\_groupe(recherche)} affiche tous les groupes correspondants à la chaîne recherchée.
                     \item \emph{recuperer\_musique(recherche)} affiche toutes les musiques correspondantes à la chaîne recherchée.
                 \end{itemize}
             \end{paragraphe}

            \begin{paragraphe}
                 Lorsqu'un résultat est sélectionné, pour l'afficher il va falloir utiliser les fonctions :
                \begin{itemize}
                    \item \emph{get\_album(album)} si un album a été sélectionné.
                    \item \emph{get\_artiste(artiste)} si un artiste a été sélectionné.
                    \item \emph{get\_groupe(groupe)} si un groupe a été sélectionné.
                    \item \mph{get\_musique(musique)} si une musique a été sélectionnée.
                \end{itemize}
            \end{paragraphe}

			\begin{paragraphe}
				\textbf{Conception IHM :}
			\end{paragraphe}

			\begin{paragraphe}
			    Nous avions, lors de la conception initiale, prévu d'afficher les résultats sous forme d'un tableau générique contenant 3 colonnes :
				\begin{itemize}
					\item Nom : le nom des albums, morceaux de musique, groupes ou artistes correspondants à la recherche entrée
					\item Type : type de la recherche : album, morceau, groupe ou artiste
					\item Date : affiche, selon le cas, la date de sortie de la musique ou de l'album, la date de naissance de l'artiste ou la date de formation du groupe.
				\end{itemize}
				Chaque résultat de la colonne << Nom >> devait être cliquable, et sur clic, diriger vers la page dédiée au résultat (i.e : la page de l'album, du morceau, du groupe ou de l'artiste correspondant).
			\end{paragraphe}
            
			\image{0.35}{page_resultat_recherche}{Mockup - Page résultats finale}

            \begin{paragraphe}
                \textbf{Si} une recherche est entrée\\
                \textbf{Alors} affichage les résultats correspondants sous forme de liens prévisionnels cliquables\\
                \textbf{Sinon} affiche tous les résultats présents dans la base de données.
            \end{paragraphe}

            \begin{paragraphe}
                \textbf{Si} clic sur un résultat\\
                \textbf{Alors} redirection vers la page de la recherche souhaitée en passant en paramètre l'id souhaité.
            \end{paragraphe}

            \begin{paragraphe}
                Nous avons cependant opté, lors du développement, sur affichage en ligne qui nous semblait plus lisible sur les différentes plateformes.\\
                En voici un exemple:
            \end{paragraphe}
            
            \image{0.33}{page_resultat_recherche_finale}{Pages publiques - Page résultat finale}

        \clearpage

		\subsubsection{Page description artiste/groupe}
            \begin{paragraphe}
                Cette page fait des requêtes sur la base de données, sur les entités << Artiste >> ou << Groupe >>.
                 en passant en paramètre le nom transmis par la page resultat de recherche. Elle affiche ensuite ces dernières.
            \end{paragraphe}

            \begin{paragraphe}
                \textbf{Service PHP :}
            \end{paragraphe}

            \begin{paragraphe}
                Pour afficher les différentes métadonnées de l'artiste ou du groupe nous avons besoin des services suivant:
                \begin{itemize}
                        \item \emph{get\_artiste(artiste)} si un artiste a été sélectionné, pour obtenir les informations de l'artiste.
                        \item \emph{get\_groupe(groupe)} si un groupe a été sélectionné, pour obtenir les informations du groupe.
                \end{itemize}
            \end{paragraphe}

			\begin{paragraphe}
				\textbf{Conception IHM :}
			\end{paragraphe}

			\begin{paragraphe}
				Initialement : \\
				On affiche tous les albums et tous les singles de l'artiste. \\
				Avec la fonction \emph{get\_album(artiste), \emph{get\_single(artiste)}.
			\end{paragraphe}

			\begin{paragraphe}
				\textbf{Si} clic sur le nom d'un album \\
				\textbf{Alors} redirection vers la page \underline{pageAlbum.php} en passant en paramètre le nom de l'album.
			\end{paragraphe}

			\image{0.35}{page_artiste}{Mockup - Page artiste/groupe}

         \newpage

            \begin{paragraphe}
                La page finale est assez fidèle au modèle que nous avions conceptualisé.
            \end{paragraphe}

            \begin{paragraphe}
                \textbf{Voici un exemple pour un artiste :}
            \end{paragraphe}
            
            \image{0.35}{page_artiste_finale2}{Pages publiques - Page artiste finale}

            \begin{paragraphe}
                \textbf{Voici un autre exemple pour un groupe :}
            \end{paragraphe}
            
            \image{0.35}{page_groupe_finale2}{Pages publiques - Page groupe finale}

        \clearpage

		\subsubsection{Page description album/musique}

			\begin{paragraphe}
                Cette page fait des requêtes sur la base de données, sur les entités << album >> ou << musique >>.
                 en passant en paramètre le nom transmis par la page resultat de recherche. Elle affiche ensuite ces dernières.
				Si un utilisateur non connecté souhaite rédiger un commenter et donner une note à un album ou à une musique,
				 il faut d'abord qu'il se connecte. Il y a donc dans l'espace commentaire un lien << connectez-vous >> le redirigeant vers la page de \underline{connexion.php}.
			\end{paragraphe}

           \begin{paragraphe}
                \textbf{Service PHP :}
            \end{paragraphe}

            \begin{paragraphe}
                Pour afficher les différentes métadonnées de l'artiste ou du groupe nous avons besoin des services suivant:
                \begin{itemize}
                        \item \emph{get\_album(album)} si un album a été sélectionné, pour obtenir les informations de l'album.
                        \item \emph{get\_musique(musique)} si une musique a été sélectionnée, pour obtenir les informations de la musique.
                \end{itemize}
            \end{paragraphe}

            \begin{paragraphe}
                \textbf{Conception IHM :}
            \end{paragraphe}

			\begin{paragraphe}
				Sur cette page, l'utilisateur pourra mettre un commentaire sur l'album ou la musique dans le cadre consacré au commentaire.
				Dans cet espace, il pourra aussi donner une note et cliquer sur le bouton << Evaluer >> afin que soit pris en compte son commentaire.
			\end{paragraphe}

			\begin{paragraphe}
				\textbf{Si} clic sur sur une des 5 étoiles \\
				\textbf{Alors} la note est définit par le numéro de l'étoile.
			\end{paragraphe}

			\begin{paragraphe}
				\textbf{Si} clic sur le bouton << evaluer >> \\
				\textbf{Alors} envoi commentaire et la note à la base de données et affichage à l'écran du commentaire et de la note visible par tout utilisateur du site.
			\end{paragraphe}

            \image{0.3}{page_album_non_connecter}{Mockup - Page album non connecté}

        \clearpage

            \image{0.3}{page_album_connecter}{Mockup - Page album connecté}

        \clearpage
        
			\begin{paragraphe}
                Les pages finales d'album sont ici aussi très ressemblantes avec notre conception.
			\end{paragraphe}

			\begin{paragraphe}
                \textbf{Voici un exemple pour album avec un utilisateur connecté :}
			\end{paragraphe}
            
            \image{0.33}{page_album_deconnecte2}{Pages publiques - Page album non connecté finale}

			\begin{paragraphe}
                \textbf{Voici le même exemple avec cette fois un utilisateur déconnecté :}
			\end{paragraphe}
            
            \image{0.33}{page_album_connecte2}{Pages publiques - Page album connecté finale}

	\clearpage

	\subsection{Pages utilisateurs}
      \begin{paragraphe}
            Les pages utilisateurs sont relatives à la gestion de compte utilisateur.
        \end{paragraphe}

		\subsubsection{Page de connexion}

			\begin{paragraphe}
				Cette page est accessible à toute personne non connectée à l'application web.
			\end{paragraphe}

			\begin{paragraphe}
				\textbf{Service PHP :}
			\end{paragraphe}

			\begin{paragraphe}
			    Cette page intéragie avec la base de données avec la fonction :
				\begin{itemize}
					\item \emph{connexion\_utilisateur(utilisateur)} : vérifie si un utilisateur correspondant existe dans la base de données.
				\end{itemize}
			\end{paragraphe}

			\begin{paragraphe}
				\textbf{Conception IHM :}
			\end{paragraphe}

			\begin{paragraphe}
				\textbf{Si} l'utilisateur n'est pas connecté \\
				\textbf{Alors} initialement la page est constituée d'un simple formulaire non rempli avec deux champs: nom utilisateur et mot de passe.
			\end{paragraphe}

            \begin{center}
                \begin{tabular}{l c | c l}
                    \textbf{Si} clic sur le bouton << connexion >> & & & \textbf{Si} clic sur lien  << s'inscrire >> \\
                    \textbf{Alors} effectue l'action \emph{connexion\_utilisateur(utilisateur).} & & & \textbf{Alors} redirection vers page inscription.
                \end{tabular}
            \end{center}

            \begin{center}
                \begin{tabular}{l}
                    \textbf{Si} l'action précédente a réussi \\
                    \textbf{Alors} l'utilisateur est ensuite redirigé vers la page principale de l'application web. \\
                    \textbf{Sinon} l'utilisateur reste sur la page avec un message d'erreur.
                \end{tabular}
            \end{center}

			\image{0.25}{page_connexion}{Mockup - Page connexion}

        \newpage
        
            \begin{paragraphe}
                Un nouveau lien a été introduit permettant à un utilisateur, ne possédant pas de compte, d'être dirigé vers le formulaire d'inscription
            \end{paragraphe}

            \image{0.60}{page_connexion_finale}{Pages Utilisateurs - Page connexion finale}

    \clearpage

		\subsubsection{Page d'inscription}

			\begin{paragraphe}
				Cette page est accessible à toutes personnes non connectées à l'application web.
			\end{paragraphe}

			\begin{paragraphe}
				\textbf{Service PHP :}
			\end{paragraphe}

			\begin{paragraphe}
			    Cette page intéragie avec la base de données avec la fonction:
				\begin{itemize}
					\item \emph{ajouter\_utilisateur(utilisateur)} : ajoute un utilisateur dans la base de données.
				\end{itemize}
			\end{paragraphe}

			\begin{paragraphe}
				\textbf{Conception IHM :}
			\end{paragraphe}

			\begin{paragraphe}
				\textbf{Si} l'utilisateur n'est pas connecté \\
				\textbf{Alors} initialement la page est constituée d'un simple formulaire non rempli avec trois champs de saisi: nom utilisateur, mot de passe et vérification mot de passe.
				Il y également un bouton << Inscription >> ainsi qu'un lien redirigeant vers la page connexion intitulé << se connecter >>.
			\end{paragraphe}

			\begin{paragraphe}
				\textbf{Si} clic sur le bouton << inscription >> \\
				\textbf{Alors} effectue l'action \emph{ajouter\_tilisateur(utilisateur)}.
			\end{paragraphe}

           \begin{paragraphe}
                \textbf{Si} clic sur lien << se connecter >> \\
                \textbf{Alors} redirection vers la page \underline{pageConnexion.php}.
            \end{paragraphe}

			\begin{paragraphe}
				\textbf{Si} l'action précédente a réussi \\
				\textbf{Alors} l'utilisateur est ensuite redirigé vers la page \underline{pageConnexion.php} avec le message : (votre compte à bien été enregistré). \\
				\textbf{Sinon} l'utilisateur reste sur la page avec un message d'erreur.
			\end{paragraphe}

            \image{0.3}{page_inscription}{Mockup - Page inscription}

        \newpage
        
            \begin{paragraphe}
                Nous avons de nouveau introduit un lien, permettant à un utilisateur possédant un compte d'être redirigé sur la page \underline{pageConnexion.php}.
            \end{paragraphe}

            \image{0.60}{page_inscription_finale}{Pages Utilisateurs - Page inscription finale}

    \clearpage

		\subsubsection{Page de gestion compte utilisateur}

			\begin{paragraphe}
				Ce formulaire est composé de 2 champs :
				\begin{itemize}
					\item Un champ pour entrer le nouveau mot de passe du compte.
					\item Un champ pour confirmer le nouveau mot de passe du compte (et ainsi éviter les maladresses de saisie).
				\end{itemize}
			\end{paragraphe}

			\begin{paragraphe}
				Après vérification par le biais d'une requête sur la base de données, la page affichera un message indiquant, si toutes les données entrée sont valides, que la modification à bien été effectuée. Ou, si certaines informations sont erronées, la nature de l'erreur et proposera de remplir à nouveau le formulaire avec de bonnes valeurs.
			\end{paragraphe}

			\begin{paragraphe}
				\textbf{Service PHP :}
			\end{paragraphe}

			\begin{paragraphe}
				Les services sont ici utilisés pour la vérification des deux mots de passes saisis. \\
				Une fois cette vérification établie, la base de données est modifiée par la nouvelle valeur correspondant au nom de compte connecté.
			\end{paragraphe}

			\begin{paragraphe}
                Il faut vérifier ici les deux mots de passes saisie. \\
                Une fois cette vérification établie, la base de données est modifiée par la nouvelle valeur correspondant au nom de compte connecté.
				On effectue les vérifications de mot de passe avec la fonction :
				\begin{itemize}
					\item \emph{modofierMotDePasse(MotDePasse)} : Change le mot de passe d'un utilisateur (celui connecté).
				\end{itemize}
			\end{paragraphe}

			\begin{paragraphe}
				\textbf{Conception IHM :}
			\end{paragraphe}

			\begin{paragraphe}
				Initalement, le formulaire est vide, il sera soumis à la page formUtilisateur.php pour la vérification.
			\end{paragraphe}

			\image{0.3}{page_compte}{Mockup - Page compte}

			\begin{paragraphe}
				\textbf{Si} clic sur le bouton << Changer de mot de passe >> \\
				\textbf{Alors} effectue l'action \emph{modifierMotDePasse(MotDePasse)}.
			\end{paragraphe}

        \newpage
        
            \begin{paragraphe}
               Au cours du développment, l'aspect et la logique de cette page a été conservé. Nous avons cependant ajouté une autre fonctionnalité
               à la gestion de compte. L'utilisateur peut supprimer son compte. Nous avons ajouté une page intermédiaire accessible en cliquant sur le bouton
               << Compte >> présent sur la bandeau d'en tête.
            \end{paragraphe}
               
            \begin{paragraphe}
                Cette page nous indique notre nom d'utilsateur et ainsi que deux boutons, l'un supprime le compte auquel nous somme connecté en transmettant une requête de suppression dans la
                base de données. Le second bouton renvoie l'utilisateur à au formulaire de modification de mot de passe.
            \end{paragraphe}

            \begin{paragraphe}
                \textbf{Voici une illustration de ces deux pages finales :}
            \end{paragraphe}

            \image{0.60}{page_compte_finale}{Pages Utilisateurs - Page gestion utilisateur finale}

            \image{0.60}{page_modif_mdp}{Pages Utilisateurs - Page modification mot de passe finale}

	\clearpage

	\subsection{Pages d'administration}

        \begin{paragraphe}
            L'accés à toutes les pages d'administration requiert d'être connecter à un compte et de posséder le statut administrateur. Elles héritent toutes de la mise en page 3.
        \end{paragraphe}
        
        \begin{paragraphe}
            Dans la version une, du dossier de conception, toutes les pages d'administrations été composées de trop de formulaires, d'actions et de boutons.
            Lors de la réalisation de l'application nous avons effectué une simplification de toutes les pages, en répartisant et délimitant les périmétres d'action de chaque page.
        \end{paragraphe}
    
        \begin{paragraphe}
            Nous avons donc délimité six catégories pour gérer les différentes entités de la base de donnée :
            \vspace{1em}
            \begin{itemize}
                \item Gestion des utilisateurs
                \item Gestion des artistes
                \item Gestion des groupes
                \item Gestion des albums
                \item Gestion des morceaux de musiques
                \item Gestion des récompenses
            \end{itemize}
        \end{paragraphe}
        
        \begin{paragraphe}
            Chaques pages des six catégories est rassemblé dans un dossier portant le nom de l'entité qu'il gère. \\
            De plus une page gestion, une page formulaire et une page action pour chaque catégorie permettent de séparer le code pour une meilleur lisibilité.
        \end{paragraphe}

        \begin{paragraphe}
            \textbf{Les pages gestions} permettent de gérer les valeurs stocké dans la base de donnée, en les modifiant ou en les supprimant.
        \end{paragraphe}
        
        \begin{paragraphe}
            \textbf{Les pages formulaires} permettent d'ajouter ou de modifier les données stocker des entités.
        \end{paragraphe}
        
        \begin{paragraphe}
            \textbf{Les pages actions} permettent quand à elles de d'effectuer des vérifications, des validations et d'exécuter les actions souhaitées.
        \end{paragraphe}

		\subsubsection{Page d'administration accueil}

			\begin{paragraphe}
				\textbf{Service PHP :}
			\end{paragraphe}

			\begin{paragraphe}
				Dans cette page, il n'y a pas de service PHP rendu. Il y a seulement des redirections vers d'autres pages.
			\end{paragraphe}

			\begin{paragraphe}
                On a supprimer les deux boutons d'ajout du genre et de récompense et seulement ajouter un bouton pour la gestion des récompenses.
			\end{paragraphe}
            
			\begin{paragraphe}
				\textbf{Conception IHM :}
			\end{paragraphe}

			\begin{paragraphe}
				Cette page hérite de la disposition n°3. \par
				Initialement cette page contient sept boutons :
				\begin{itemize}
					\item Gestion artistes.
					\item Gestion groupes.
					\item Gestion albums.
					\item Gestion musiques.
					\item Gestion utilisateurs.
                    \item Gestion récompense.
				\end{itemize}
			\end{paragraphe}

            \begin{center}
                \begin{tabular}{l | l}
                    \textbf{Si} clic sur le bouton << Gestion artistes >> & \textbf{Si} clic sur le bouton << Gestion groupes >> \\
                    \textbf{Alors} redirection vers adminGestionArtiste.php. & \textbf{Alors} redirection vers adminGestionGroupe.php. \\ \\
                    
                    \textbf{Si} clic sur le bouton << Gestion albums >> & \textbf{Si} clic sur le bouton << Gestion musiques >> \\
                    \textbf{Alors} redirection vers admingGestionAlbum.php. & \textbf{Alors} redirection vers adminGestionMusique.php. \\ \\
                    
                    \textbf{Si} clic sur le bouton << Gestion utilisateur >> & \textbf{Si} clic sur le bouton << Gestion récompense >> \\
                    \textbf{Alors} redirection vers adminGestionutilisateur.php. & \textbf{Alors} redirection vers adminGestionRecompense.php.
                \end{tabular}
            \end{center}

        \clearpage

            \begin{paragraphe}
                \textbf{Voici le Mockup prévu :}
            \end{paragraphe}

			\image{0.35}{page_admin_index}{Mockup - Page d'administration accueil}
            
            \begin{paragraphe}
                \textbf{Voici le nouveau panneau d'administration :}
            \end{paragraphe}
            
			\image{0.35}{page_admin_index_final}{Page administration - Accueil final}
            
    \clearpage

		\subsubsection{Page gestion artiste}
        
            \begin{paragraphe}
                La gestion des artistes se fais avec trois pages, une page de gestion, une page de formulaire et une page d'action.
            \end{paragraphe}

			\begin{paragraphe}
				\textbf{Service PHP :}
			\end{paragraphe}

			\begin{paragraphe}
                La page de gestion et la page formulaire permettent d'envoyer des actions à effectué sur la page action.
			\end{paragraphe}

			\begin{paragraphe}
                \textbf{Voici donc la liste des actions possible qu'effectue la page \underline{actionArtiste.php} :}
            
                \begin{itemize}
                    \item \emph{ajouter\_artiste(artiste)} :
                \end{itemize}
                \begin{paragraphe}
                    Permet de vérifier si les informations renseigner sont correcte. Le nom, le prénom et la date de naissance de l'artiste étant obligatoire. Ajoute ensuite les informations dans la base de données.
                \end{paragraphe}
                
                \begin{itemize}
                    \item \emph{modifier\_artiste(artiste)} :
                \end{itemize}
                \begin{paragraphe}
                    Permet de vérifier si les informations renseigner sont correcte. Le nom, le prénom et la date de naissance de l'artiste étant obligatoire. Modifie ensuite les informations dans la base de données.
                \end{paragraphe}
                
                \begin{itemize}
                    \item \emph{supprime\_artiste(artiste)} :
                \end{itemize}
                \begin{paragraphe}
                    Permet de supprimer un artiste dans la base de données.
                \end{paragraphe}
            \end{paragraphe}

			\begin{paragraphe}
				\textbf{Conception IHM :}
			\end{paragraphe}
            
            \begin{paragraphe}
                \textbf{Page de gestion :} elle récupère toutes les entrées des artistes à l'aide de la fonction \emph{getListeArtiste()} pour les afficher dans un tableau.
            \end{paragraphe}

            \begin{center}
                \begin{tabular}{l | l}
                    \textbf{Si} clic sur le titre << Panneau administration >> & \textbf{Si} clic sur le bouton << Gestion artiste >> \\
                    \textbf{Alors} redirection vers \underline{adminAccueil.php}. & \textbf{Alors} redirection vers \underline{gestionArtiste.php}. \\ \\

                    \textbf{Si} clic sur le bouton << Ajouter artiste >> & \textbf{Si} clic sur le bouton << Modifier >> \\
                    \textbf{Alors} redirection vers \underline{formArtiste.php}. & \textbf{Alors} redirection vers \underline{formArtiste.php}. \\ \\
                \end{tabular}

                \begin{tabular}{l}
                    \textbf{Si} clic sur le bouton << Supprimer >> \\
                    \textbf{Alors} effectue l'action \emph{supprimer\_artiste(artiste)}.
                \end{tabular}
            \end{center}
            
			\image{0.4}{page_admin_gestion_artiste_final}{Page administration - Gestion artistes}
            
            \begin{paragraphe}
                \textbf{Page du formulaire d'un artiste :}
            \end{paragraphe}
            
            \begin{paragraphe}
                La page est composé de champs pour chaque élément d'un artiste (définit dans le cahier des charges).
            \end{paragraphe}

            \begin{paragraphe}
                \textbf{Si} un identifiant d'un artiste existant n'est pas renseigné à la page \\
                \textbf{Alors} on laisse les champs du formulaire vide, il s'agit de crée un nouvel artiste. \\
                \textbf{Sinon} il s'agit de modifier un artiste. on récupère les données de l'artiste avec la fonction \emph{getArtiste(artiste)}.
            \end{paragraphe}
            
        \newpage
            
            \begin{center}
                \begin{tabular}{l | l}
                    \textbf{Si} clic sur le bouton << Ajouter >> & \textbf{Si} clic sur le bouton << Modifier >> \\
                    \textbf{Alors} effectuer l'action \emph{ajouterArtiste(artiste)}. & \textbf{Alors} effectue l'action \emph{modifierArtiste(artiste)}.
                \end{tabular}
            \end{center}
            
			\image{0.35}{page_admin_formulaire_artiste_ajouter_final}{Page administration - Formulaire artistes}
			
            \image{0.35}{page_admin_formulaire_artiste_modifier_final}{Page administration - Formulaire artistes}

    \clearpage

		\subsubsection{Page gestion groupe}

			\begin{paragraphe}
                La page gestion reprend le même principe que précédement. 
			\end{paragraphe}

			\begin{paragraphe}
				\textbf{Service PHP :}
			\end{paragraphe}
            
            \begin{paragraphe}
                La page de gestion et la page formulaire permettent d'envoyer des actions à effectué sur la page action.
			\end{paragraphe}

			\begin{paragraphe}
                \textbf{Voici donc la liste des actions possible qu'effectue la page \underline{actionGroupe.php} :}
            
                \begin{itemize}
                    \item \emph{ajouter\_groupe(groupe)} :
                \end{itemize}
                \begin{paragraphe}
                    Permet de vérifier si les informations renseigner sont correcte. Le nom, la date de création du groupe sont obligatoire obligatoire. De plus il est obligatoire de sélectionner deux artistes au minimum. Ajoute ensuite les informations dans la base de données.
                \end{paragraphe}
                
                \begin{itemize}
                    \item \emph{modifier\_groupe(groupe)} :
                \end{itemize}
                \begin{paragraphe}
                    Permet de vérifier si les informations renseigner sont correcte. Le nom, la date de création du groupe sont obligatoire obligatoire. De plus il est obligatoire de sélectionner deux artistes au minimum. Modifie ensuite les informations dans la base de données.
                \end{paragraphe}
                
                \begin{itemize}
                    \item \emph{supprime\_groupe(groupe)} :
                \end{itemize}
                \begin{paragraphe}
                    Permet de supprimer un groupe dans la base de données.
                \end{paragraphe}
            \end{paragraphe}

			\begin{paragraphe}
				\textbf{Conception IHM :}
			\end{paragraphe}
            
            \begin{paragraphe}
                \textbf{Page de gestion :} elle récupère toutes les entrées des groupes à l'aide de la fonction \emph{getListeGroupe()} pour les afficher dans un tableau.
            \end{paragraphe}

            \begin{center}
                \begin{tabular}{l | l}
                    \textbf{Si} clic sur le titre << Panneau administration >> & \textbf{Si} clic sur le bouton << Gestion groupe >> \\
                    \textbf{Alors} redirection vers \underline{adminAccueil.php}. & \textbf{Alors} redirection vers \underline{gestionGroupe.php}. \\ \\

                    \textbf{Si} clic sur le bouton << Ajouter groupe >> & \textbf{Si} clic sur le bouton << Modifier >> \\
                    \textbf{Alors} redirection vers \underline{formGroupe.php}. & \textbf{Alors} redirection vers \underline{formGroupe.php}. \\ \\
                \end{tabular}

                \begin{tabular}{l}
                    \textbf{Si} clic sur le bouton << Supprimer >> \\
                    \textbf{Alors} effectue l'action \emph{supprimer\_groupe(groupe)}.
                \end{tabular}
            \end{center}
            
			\image{0.4}{page_admin_gestion_groupe_final}{Page administration - Gestion groupe}
            
        \clearpage
        
            \begin{paragraphe}
                \textbf{Page du formulaire d'un groupe :}
            \end{paragraphe}
            
            \begin{paragraphe}
                La page est composé de champs pour chaque élément d'un artiste (définit dans le cahier des charges). \\
                Ainsi qu'une liste de case à coché des artistes présent dans la base de données.
            \end{paragraphe}

            \begin{paragraphe}
                \textbf{Si} un identifiant d'un groupe existant n'est pas renseigné à la page \\
                \textbf{Alors} on laisse les champs du formulaire vide, il s'agit de crée un nouveau groupe. \\
                \textbf{Sinon} il s'agit de modifier un groupe. On récupère les données du groupe avec la fonction \emph{getGroupe(groupe)} ainsi que l'association avec les artistes du groupe.
            \end{paragraphe}
            
            \begin{center}
                \begin{tabular}{l | l}
                    \textbf{Si} clic sur le bouton << Ajouter >> & \textbf{Si} clic sur le bouton << Modifier >> \\
                    \textbf{Alors} effectuer l'action \emph{ajouterGroupe(groupe)}. & \textbf{Alors} effectue l'action \emph{modifierGroupe(groupe)}.
                \end{tabular}
            \end{center}
            
			\image{0.35}{page_admin_formulaire_groupe_ajouter_final}{Page administration - Formulaire groupe}
			
            \image{0.35}{page_admin_formulaire_groupe_modifier_final}{Page administration - Formulaire groupe}
            
    \newpage

		\subsubsection{Page gestion album}

			\begin{paragraphe}
                La page gestion reprend le même principe que précédement. On a rajouter aussi des pages pour associer des musiques à un album.
			\end{paragraphe}

			\begin{paragraphe}
				\textbf{Service PHP :}
			\end{paragraphe}
            
            \begin{paragraphe}
                La page de gestion et la page formulaire permettent d'envoyer des actions à effectué sur la page action.
			\end{paragraphe}

			\begin{paragraphe}
                \textbf{Voici la liste des actions possible qu'effectue la page \underline{actionAlbum.php} :}
            
                \begin{itemize}
                    \item \emph{ajouter\_album(album)} :
                \end{itemize}
                \begin{paragraphe}
                    Permet de vérifier si les informations renseigner sont correcte. Le nom, la date de création de l'album sont obligatoire. De plus il est obligatoire de sélectionner un artiste ou un groupe de musique au minimum. Ajoute ensuite les informations dans la base de données.
                \end{paragraphe}
                
                \begin{itemize}
                    \item \emph{modifier\_album(album)} :
                \end{itemize}
                \begin{paragraphe}
                    Permet de vérifier si les informations renseigner sont correcte. Le nom, la date de création de l'album sont obligatoire. De plus il est obligatoire de sélectionner un artiste ou un groupe de musique au minimum. Modifie ensuite les informations dans la base de données.
                \end{paragraphe}
                
                \begin{itemize}
                    \item \emph{supprime\_album(album)} :
                \end{itemize}
                \begin{paragraphe}
                    Permet de supprimer un album dans la base de données.
                \end{paragraphe}
            \end{paragraphe}
            
            \begin{paragraphe}
                De plus d'autres pages permettent d'associé une musique avec l'album.
            \end{paragraphe}

			\begin{paragraphe}
                \textbf{Voici la liste des actions possible qu'effectue la page \underline{actionAssemblerAlbum.php} :}
            
                \begin{itemize}
                    \item \emph{ajouter\_musique\_album(album, musique)} :
                \end{itemize}
                \begin{paragraphe}
                    Permet de vérifier si les informations renseigner sont correcte. Un numéro de piste pour la musique dans l'album est obligatoire. Ajoute ensuite les informations dans la base de données.
                \end{paragraphe}
                
                \begin{itemize}
                    \item \emph{modifier\_musique\_album(album, musique)} :
                \end{itemize}
                \begin{paragraphe}
                    Permet de vérifier si les informations renseigner sont correcte. Un numéro de piste pour la musique dans l'album est obligatoire. Modifie ensuite les informations dans la base de données.
                \end{paragraphe}
                
                \begin{itemize}
                    \item \emph{supprime\_musique\_album(album, musique)} :
                \end{itemize}
                \begin{paragraphe}
                    Permet de supprimer un album dans la base de données.
                \end{paragraphe}
            \end{paragraphe}

			\begin{paragraphe}
				\textbf{Conception IHM :}
			\end{paragraphe}
            
            \begin{paragraphe}
                \textbf{Page de gestion :} elle récupère toutes les entrées des albums à l'aide de la fonction \emph{getListeAlbum()} pour les afficher dans un tableau.
            \end{paragraphe}

            \begin{center}
                \begin{tabular}{l | l}
                    \textbf{Si} clic sur le titre << Panneau administration >> & \textbf{Si} clic sur le bouton << Gestion album >> \\
                    \textbf{Alors} redirection vers \underline{adminAccueil.php}. & \textbf{Alors} redirection vers \underline{gestionArtiste.php}. \\ \\

                    \textbf{Si} clic sur le bouton << Ajouter album >> & \textbf{Si} clic sur le bouton << Modifier >> \\
                    \textbf{Alors} redirection vers \underline{formAlbum.php}. & \textbf{Alors} redirection vers \underline{formAlbum.php}. \\ \\
                    
                    \textbf{Si} clic sur le bouton << Gestion musique >> & \textbf{Si} clic sur le bouton << Supprimer >> \\
                    \textbf{Alors} redirection vers \underline{gestionAssemblerAlbum.php}. & \textbf{Alors} effectue l'action \emph{supprimer\_album(album)}.
                \end{tabular}
            \end{center}

        \clearpage
            
			\image{0.4}{page_admin_gestion_album_final}{Page administration - Gestion album}
        
            \begin{paragraphe}
                \textbf{Page du formulaire d'un album :}
            \end{paragraphe}
            
            \begin{paragraphe}
                La page est composé de champs pour chaque élément d'un album (définit dans le cahier des charges). \\
                Ainsi qu'une liste de case à coché des artistes ou des groupes présent dans la base de données.
            \end{paragraphe}

            \begin{paragraphe}
                \textbf{Si} un identifiant d'un album existant n'est pas renseigné à la page \\
                \textbf{Alors} on laisse les champs du formulaire vide, il s'agit de crée un nouveau album. \\
                \textbf{Sinon} il s'agit de modifier un album. On récupère les données de l'album avec la fonction \emph{getAlbum(album)} ainsi que l'association avec les artistes ou groupes de l'album.
            \end{paragraphe}
            
            \begin{center}
                \begin{tabular}{l | l}
                    \textbf{Si} clic sur le bouton << Ajouter >> & \textbf{Si} clic sur le bouton << Modifier >> \\
                    \textbf{Alors} effectuer l'action \emph{ajouterAlbum(album)}. & \textbf{Alors} effectue l'action \emph{modifierAlbum(album)}.
                \end{tabular}
            \end{center}
            
			\image{0.35}{page_admin_formulaire_album_modifier_final}{Page administration - Formulaire album}
            
        \newpage
            
            \begin{paragraphe}
                \textbf{Page de gestion des musique d'un album :} elle récupère toutes les musiques associés à l'album, passé en paramètre, à l'aide de la fonction \emph{getListeMusiqueAlbum()} pour les afficher dans un tableau.
            \end{paragraphe}

            \begin{center}
                \begin{tabular}{l | l}
                    \textbf{Si} clic sur le titre << Panneau administration >> & \textbf{Si} clic sur le bouton << Gestion album >> \\
                    \textbf{Alors} redirection vers \underline{adminAccueil.php}. & \textbf{Alors} redirection vers \underline{gestionArtiste.php}. \\ \\

                    \textbf{Si} clic sur le bouton << Gérer l'album album >> & \textbf{Si} clic sur le bouton << Ajouter musique >> \\
                    \textbf{Alors} redirection vers \underline{formAssemblerAlbum.php}. & \textbf{Alors} redirection vers \underline{formAssemblerAlbum.php}. \\ \\
                    
                    \textbf{Si} clic sur le bouton << Modifier >> & \textbf{Si} clic sur le bouton << Supprimer >> \\
                    \textbf{Alors} redirection vers \underline{formAssemblerAlbum.php}. & \textbf{Alors} effectue l'action \emph{supprimer\_musique\_album(album, musique)}.
                \end{tabular}
            \end{center}
            
			\image{0.35}{page_admin_gestion_assembler_album_final}{Page administration - Gestion musique d'un album}
        
            \begin{paragraphe}
                \textbf{Page du formulaire musique d'un album :}
            \end{paragraphe}
            
            \begin{paragraphe}
                La page est composé du numéro de piste. \\
                De plus s'il s'agit d'un ajout d'un nouveau morceau de musique à l'album alors une liste de case à coché est disponible correspondant aux des musiques des artistes ou des groupes dont l'album est composé.
            \end{paragraphe}
            
            \begin{center}
                \begin{tabular}{l}
                    \textbf{Si} clic sur le bouton << Ajouter >> \\
                    \textbf{Alors} effectuer l'action \emph{ajouterAssemblerAlbum(album, musique)}.
                \end{tabular}
            \end{center}
            
			\image{0.35}{page_admin_formulaire_assembler_album_ajouter_final}{Page administration - Formulaire ajouter musique album}
            
        \clearpage
            
            \begin{center}
                \begin{tabular}{l}
                    \textbf{Si} clic sur le bouton << Modifier >> \\
                    \textbf{Alors} effectue l'action \emph{modifierAssemblerAlbum(album, musique)}.
                \end{tabular}
            \end{center}
            
			\image{0.35}{page_admin_formulaire_assembler_album_modifier_final}{Page administration - Formulaire modifier musique album}
        
    \newpage

		\subsubsection{Page gestion musique}

			\begin{paragraphe}
                La page gestion reprend le même principe que précédement. \\
                La page a été allégé des formulaires trop compliqué et trop chargé.
			\end{paragraphe}

			\begin{paragraphe}
				\textbf{Service PHP :}
			\end{paragraphe}
            
            \begin{paragraphe}
                La page de gestion et la page formulaire permettent d'envoyer des actions à effectué sur la page action.
			\end{paragraphe}
            
            \begin{paragraphe}
                \textbf{Voici la liste des actions possible qu'effectue la page \underline{actionMusique.php} :}
            
                \begin{itemize}
                    \item \emph{ajouter\_musique(musique)} :
                \end{itemize}
                \begin{paragraphe}
                    Permet de vérifier si les informations renseigner sont correcte.
                    Le titre, la durée ainsi que la date de création du morceau de musique sont obligatoire.
                    De plus il est obligatoire de sélectionner un artiste ou un groupe de musique au minimum. L'ajout d'un genre est facultatif.
                    Ajoute ensuite les informations dans la base de données.
                \end{paragraphe}
                
                \begin{itemize}
                    \item \emph{modifier\_musique(musique)} :
                \end{itemize}
                \begin{paragraphe}
                    Permet de vérifier si les informations renseigner sont correcte.
                    Le titre, la durée ainsi que la date de création du morceau de musique sont obligatoire.
                    De plus il est obligatoire de sélectionner un artiste ou un groupe de musique au minimum. L'ajout d'un genre est facultatif.
                    Modifie ensuite les informations dans la base de données.
                \end{paragraphe}
                
                \begin{itemize}
                    \item \emph{supprime\_musique(musique)} :
                \end{itemize}
                \begin{paragraphe}
                    Permet de supprimer une musique dans la base de données.
                \end{paragraphe}
            \end{paragraphe}

			\begin{paragraphe}
				\textbf{Conception IHM :}
			\end{paragraphe}
            
            \begin{paragraphe}
                \textbf{Page de gestion :} elle récupère toutes les entrées des morceau de musique à l'aide de la fonction \emph{getListeMusique()} pour les afficher dans un tableau.
            \end{paragraphe}

            \begin{center}
                \begin{tabular}{l | l}
                    \textbf{Si} clic sur le titre << Panneau administration >> & \textbf{Si} clic sur le bouton << Gestion recompense >> \\
                    \textbf{Alors} redirection vers \underline{adminAccueil.php}. & \textbf{Alors} redirection vers \underline{gestionRecompense.php}. \\ \\

                    \textbf{Si} clic sur le bouton << Ajouter musique >> & \textbf{Si} clic sur le bouton << Modifier >> \\
                    \textbf{Alors} redirection vers \underline{formMusique.php}. & \textbf{Alors} redirection vers \underline{formMusique.php}. \\ \\
                \end{tabular}

                \begin{tabular}{l}
                    \textbf{Si} clic sur le bouton << Supprimer >> \\
                    \textbf{Alors} effectue l'action \emph{supprimer\_musique(musique)}.
                \end{tabular}
            \end{center}
            
			\image{0.4}{page_admin_gestion_musique_final}{Page administration - Gestion recompense}
        
            \begin{paragraphe}
                \textbf{Page du formulaire musique d'un album :}
            \end{paragraphe}
            
            \begin{paragraphe}
                La page est composé de champs pour chaque élément d'une récompense (définit dans le cahier des charges). \\
                De plus trois section de case à coché permettent de sélectionner les genres du morceau de musique. Ainsi que les artistes ou les groupes à avoir composés le morceau de musique. \\
                On doit obligatoirement renseigné un artiste ou un groupe de musique. Le genre étant optionnels.
            \end{paragraphe}
            
            \begin{center}
                \begin{tabular}{l | l}
                    \textbf{Si} clic sur le bouton << Ajouter >> & \textbf{Si} clic sur le bouton << Modifier >> \\
                    \textbf{Alors} effectuer l'action \emph{ajouterMusique(musique)}. & \textbf{Alors} effectue l'action \emph{modifierMusique(musique)}.
                \end{tabular}
            \end{center}
            
			\image{0.35}{page_admin_formulaire_musique_ajouter_final}{Page administration - Formulaire ajouter musique}
            
			\image{0.35}{page_admin_formulaire_musique_modifier_final}{Page administration - Formulaire modifier musique}







            
	\clearpage

		\subsubsection{Page gestion récompense}

			\begin{paragraphe}
                La page gestion reprend le même principe que précédement.
			\end{paragraphe}

			\begin{paragraphe}
				\textbf{Service PHP :}
			\end{paragraphe}
            
            \begin{paragraphe}
                La page de gestion et la page formulaire permettent d'envoyer des actions à effectué sur la page action.
			\end{paragraphe}

			\begin{paragraphe}
                \textbf{Voici la liste des actions possible qu'effectue la page \underline{actionRecompense.php} :}
            
                \begin{itemize}
                    \item \emph{ajouter\_recompense(recompense)} :
                \end{itemize}
                \begin{paragraphe}
                    Permet de vérifier si les informations renseigner sont correcte. Le nom et la date de la récompense sont obligatoire. De plus il est obligatoire de sélectionner un artiste au minimum. Ajoute ensuite les informations dans la base de données.
                \end{paragraphe}
                
                \begin{itemize}
                    \item \emph{modifier\_recompense(recompense)} :
                \end{itemize}
                \begin{paragraphe}
                    Permet de vérifier si les informations renseigner sont correcte. Le nom et la date de la récompense sont obligatoire. De plus il est obligatoire de sélectionner un artiste au minimum. Modifie ensuite les informations dans la base de données.
                \end{paragraphe}
                
                \begin{itemize}
                    \item \emph{supprime\_recompense(recompense)} :
                \end{itemize}
                \begin{paragraphe}
                    Permet de supprimer un album dans la base de données.
                \end{paragraphe}
            \end{paragraphe}

			\begin{paragraphe}
				\textbf{Conception IHM :}
			\end{paragraphe}
            
            \begin{paragraphe}
                \textbf{Page de gestion :} elle récupère toutes les entrées des albums à l'aide de la fonction \emph{getListeAlbum()} pour les afficher dans un tableau.
            \end{paragraphe}

            \begin{center}
                \begin{tabular}{l | l}
                    \textbf{Si} clic sur le titre << Panneau administration >> & \textbf{Si} clic sur le bouton << Gestion recompense >> \\
                    \textbf{Alors} redirection vers \underline{adminAccueil.php}. & \textbf{Alors} redirection vers \underline{gestionRecompense.php}. \\ \\

                    \textbf{Si} clic sur le bouton << Ajouter recompense >> & \textbf{Si} clic sur le bouton << Modifier >> \\
                    \textbf{Alors} redirection vers \underline{formRecompense.php}. & \textbf{Alors} redirection vers \underline{formRecompense.php}. \\ \\
                \end{tabular}

                \begin{tabular}{l}
                    \textbf{Si} clic sur le bouton << Supprimer >> \\
                    \textbf{Alors} effectue l'action \emph{supprimer\_recompense(recompense)}.
                \end{tabular}
            \end{center}
            
			\image{0.4}{page_admin_gestion_recompense_final}{Page administration - Gestion recompense}
            
        \clearpage
        
            \begin{paragraphe}
                \textbf{Page du formulaire d'une récompense :}
            \end{paragraphe}
            
            \begin{paragraphe}
                La page est composé de champs pour chaque élément d'une récompense (définit dans le cahier des charges). \\
                Ainsi qu'une liste de case à coché des artistes présent dans la base de données qui reçoivent cette récompense.
            \end{paragraphe}

            \begin{paragraphe}
                \textbf{Si} un identifiant d'une récompense existant n'est pas renseigné à la page \\
                \textbf{Alors} on laisse les champs du formulaire vide, il s'agit de crée une nouvelles récompense. \\
                \textbf{Sinon} il s'agit de modifier une récompense. On récupère les données de la récompense avec la fonction \emph{getRecompense(recompense)} ainsi que l'association avec les artistes.
            \end{paragraphe}
            
            \begin{center}
                \begin{tabular}{l}
                    \textbf{Si} clic sur le bouton << Ajouter >> \\
                    \textbf{Alors} effectuer l'action \emph{ajouter\_recompense(recompense)}.
                \end{tabular}
            \end{center}
            
			\image{0.35}{page_admin_formulaire_recompense_ajouter_final}{Page administration - Formulaire récompense}
            
            \begin{center}
                \begin{tabular}{l}
                    \textbf{Si} clic sur le bouton << Modifier >> \\
                    \textbf{Alors} effectue l'action \emph{modifier\_recompense(recompense)}.
                \end{tabular}
            \end{center}
            
			\image{0.35}{page_admin_formulaire_recompense_modifier_final}{Page administration - Formulaire récompense}
        
        \newpage

		\subsubsection{Page gestion compte utilisateur}

			\begin{paragraphe}
                La page gestion reprend le même principe que précédement.
			\end{paragraphe}

			\begin{paragraphe}
				\textbf{Service PHP :}
			\end{paragraphe}
            
            \begin{paragraphe}
                La page de gestion et la page formulaire permettent d'envoyer des actions à effectué sur la page action.
			\end{paragraphe}

			\begin{paragraphe}
                \textbf{Voici la liste des actions possible qu'effectue la page \underline{actionUtilisateur.php} :}
            
                \begin{itemize}
                    \item \emph{ajouter\_utilisateur(utilisateur)} :
                \end{itemize}
                \begin{paragraphe}
                    Permet de vérifier si les informations renseigner sont correcte. Le nom, le mot de passe et la vérification du mot de passe sont obligatoire. Ajoute ensuite les informations dans la base de données.
                \end{paragraphe}
                
                \begin{itemize}
                    \item \emph{modifier\_motDePasse\_utilisateur(utilisateur)} :
                \end{itemize}
                \begin{paragraphe}
                    Permet d'effectuer un changement de mot de passe dans la base de données d'un utilisateur.
                \end{paragraphe}
                
                \begin{itemize}
                    \item \emph{modifier\_statut\_utilisateur(utilisateur)} :
                \end{itemize}
                \begin{paragraphe}
                    Permet de changer le statut d'un utilisateur normal en statut adminsitrateur, ou en statut d'un utilisateur normal. \\
                    Un administrateur ne peut changer son propre statut en utilisateur normal.
                \end{paragraphe}
                
                \begin{itemize}
                    \item \emph{supprimer\_utilisateur(utilisateur)} :
                \end{itemize}
                \begin{paragraphe}
                    Permet de supprimer un album dans la base de données.
                    Un administrateur ne peut supprimer son propre compte.
                \end{paragraphe}
            \end{paragraphe}

			\begin{paragraphe}
				\textbf{Conception IHM :}
			\end{paragraphe}

            \begin{paragraphe}
                \textbf{Voici le Mockup prévu :}
            \end{paragraphe}

			\image{0.35}{page_admin_compte}{Mockup - Page d'administration - Page gestion compte utilisateur}

        \clearpage
    
            \begin{paragraphe}
                On a rajouter une page formulaire en plus pour ajouter ou pour modifier le mot de passe d'un utilisateur.
            \end{paragraphe}
            
            \begin{paragraphe}
                \textbf{Voici le nouveau panneau d'administration :}
            \end{paragraphe}

            \begin{center}
                \begin{tabular}{l | l}
                    \textbf{Si} clic sur le titre << Panneau administration >> & \textbf{Si} clic sur le bouton << Gestion recompense >> \\
                    \textbf{Alors} redirection vers \underline{adminAccueil.php}. & \textbf{Alors} redirection vers \underline{gestionRecompense.php}. \\ \\

                    \textbf{Si} clic sur le bouton << Ajouter utilisateur >> & \textbf{Si} clic sur le bouton << Devenir User/Admin >> \\
                    \textbf{Alors} redirection vers \underline{formUtilisateur.php}. & \textbf{Alors} effectue l'action \emph{modifier\_statut\_utilisateur(utilisateur)}. \\ \\

                    \textbf{Si} clic sur le bouton << Modifier >> & \textbf{Si} clic sur le bouton << Supprimer >> \\
                    \textbf{Alors} redirection vers \underline{formRecompense.php}. & \textbf{Alors} effectue l'action \emph{supprimer\_utilisateur(utilisateur)}.
                \end{tabular}
            \end{center}
            
			\image{0.35}{page_admin_gestion_utilisateur_final}{Page administration - Gestion utilisateur final}
            
            \begin{paragraphe}
                \textbf{Page du formulaire d'un utilisateur :}
            \end{paragraphe}
            
            \begin{paragraphe}
                La page est composé des champs nom utilisateur, mot de passe et vérification du mot de passe saisi.
            \end{paragraphe}

            \begin{paragraphe}
                \textbf{Si} un identifiant d'un utilisateur existant n'est pas renseigné à la page \\
                \textbf{Alors} on laisse les champs du formulaire vide, il s'agit de crée un nouveau utilisateur. \\
                \textbf{Sinon} il s'agit de modifier le mot de passe d'un utilisateur. On récupère les données de l'utilsiateur avec la fonction \emph{getUtilisateur(utilisateur)}.
            \end{paragraphe}
            
            \begin{center}
                \begin{tabular}{l}
                    \textbf{Si} clic sur le bouton << Ajouter >> \\
                    \textbf{Alors} effectuer l'action \emph{ajouter\_utilisateur(utilisateur)}.
                \end{tabular}
            \end{center}
            
			\image{0.35}{page_admin_formulaire_utilisateur_ajouter_final}{Page administration - Formulaire utilisateur}
            
        \newpage
            
            \begin{center}
                \begin{tabular}{l}
                    \textbf{Si} clic sur le bouton << Modifier >> \\
                    \textbf{Alors} effectue l'action \emph{modifier\_utilisateur(utilisateur)}.
                \end{tabular}
            \end{center}
            
			\image{0.35}{page_admin_formulaire_utilisateur_modifier_final}{Page administration - Formulaire utilisateur}
