
\section{Introduction}

	\begin{paragraphe}
		Ce document a pour but de décrire les aspects visuels et techniques de notre site.\\
		Ainsi, il présente les différentes interactions entre les différentes pages que nous avons conceptualisées.
		Il montre également sous quelles dispositions peuvent être retrouvé les différents éléments de notre site web et les moyens d'y accéder.\\
		Lors de la phase de développement du site, nous sommes revenu sur certains choix que nous avions fait lors de la conception.
		Nous nous attacherons donc ici à faire apparaître les différentes évolutions en prenant soin d'expliquer les raisons de
		ces changements et la réflexion que nous avons eu.
	\end{paragraphe}

\section{Charte graphique}

	\begin{paragraphe}
	    Nous avions d'abord opté pour une charte avec un fond beige entre deux bandeaux bleu azur, mais pour des raisons
	    d'esthétisme, de clarté et de lisibilité nous avons décidé de changer ces couleurs.\\

		Notre site a maintenant pour couleurs principales:
            \begin{itemize}
            \item une couleur d'écriture blanche.
            \item une couleur orangée pour les boutons, les mots-clés et le bas de page.
            \item une couleur gris sombre pour le fond et l'en tête de page.
            \end{itemize}
    \end{paragraphe}
    \image{0.50}{charte_graphique_couleurs}{Charte graphique - Couleurs}

    \begin{paragraphe}
		Le site comprend également deux bandeaux, un en haut et un en bas de notre chaque page.\\
		Le bandeau du haut comprend un bouton <<accueil>>,
		un bouton <<connexion>>, un bouton <<inscription>> si l'on n'est pas connecté et un bouton <<À propos>> contenant des metas-informations sur le site.
        \image{0.50}{bandeau_haut}{Charte graphique - Bandeau du haut}

        Le bandeau du bas rapelle le nom du site << Critique Musicale>> ainsi que la date de dernière maintenance du site.
    \end{paragraphe}
    \image{0.60}{bandeau_bas}{Charte graphique - Bandeau du bas}

\newpage

	\begin{paragraphe}
		Une barre de recherche est également présente en dessous du bandeau du haut sur toutes les pages du site, hormis celles de connexion, de gestion de compte et d'à propos.
		Elle permet de saisir des recherches d'artiste, d'album, de groupe ou de morceau de musique.
	\end{paragraphe}
    \image{0.50}{barre_recherche}{Charte graphique - Barre de recherche}

	\begin{paragraphe}
		Le style et la taille de la police des caractères normaux pour toutes les pages sont par défaut respectivement \og hurme\_no2-webfont >> et << sans-serif >>.\\
		Les titres sont en gras de couleur orangée. Ils ont pour style le << hurme\_no2-webfont >> en << sans-serif >> avec une taille de police de 15px.
    \end{paragraphe}
    \image{0.43}{exemple_ecriture}{Charte graphique - Exemple du style d'écriture sur une page}


    \begin{paragraphe}
       Pour des affichages sous forme de tableau, nous avons choisi un design sobre, favorisant la lisibilité et la recherche rapide en alternant les couleurs des lignes
       et séparant distinctement les différentes lignes et colonnes du tableau.
    \end{paragraphe}
   \image{0.60}{tableau}{Charte graphique - Exemple d'un tableau}

\newpage

\section{Présentation}
	\subsection{Disposition 1 (layout)}

		\begin{paragraphe}
			Ceci est la première disposition dont toutes les pages vont hériter dans le cas où l'utilisateur n'est pas connecté.
		\end{paragraphe}

        \begin{center}
            \begin{tabular}{l c | c l}
                \textbf{Si} clic sur le bouton << accueil >> & & & \textbf{Si} clic sur le bouton << connexion >> \\
                \textbf{Alors} redirection vers \underline{pageAccueil.php}. & & & \textbf{Alors} redirection vers \underline{pageConnexion.php}.
            \end{tabular}
        \end{center}
        
        \begin{center}
            \begin{tabular}{l}
                \textbf{Si} clic sur bouton << inscription >> \\
                \textbf{Alors} redirection vers \underline{pageInscription.php}.
            \end{tabular}
        \end{center}

		\image{0.35}{page_type_1}{Mockup - Layout 1}

        \begin{paragraphe}
            Le layout final obtenu est très proche de ce que nous avions conceptualisé. Nous avons ajouté un bouton << À propos >>,
            déplacé le bouton << accueil >> sur la droite et affiché à sa place le titre du site qui dirige vers l'accueil lors de son activation.\\
            Nous avons également introduit la barre de recherche sur chaque page.
        \end{paragraphe}

        \image{0.30}{page_type_1_final}{Disposition 1 - Layout final 1}

    \newpage

	\subsection{Disposition 2 (layout)}

		\begin{paragraphe}
			Ceci est la seconde disposition dont toutes les pages vont hériter dans le cas où un utilisateur normal est connecté.
		\end{paragraphe}

        \begin{center}
            \begin{tabular}{l c | c l}
                \textbf{Si} clic sur le bouton << accueil >> & & & \textbf{Si} clic sur le bouton << compte >> \\
                \textbf{Alors} redirection vers \underline{pageAccueil.php}. & & & \textbf{Alors} redirection vers \underline{pageCompte.php}.
            \end{tabular}
        \end{center}
        
        \begin{center}
            \begin{tabular}{l}
                \textbf{Si} clic sur bouton << déconnexion >> \\
                \textbf{Alors} on déconnecte l'utilisateur de sa session courante.
            \end{tabular}
        \end{center}

		\image{0.35}{page_type_2}{Mockup - Layout 2}

        \begin{paragraphe}
            Le layout final obtenu est également très proche de ce que nous avions conceptualisé. Nous avons de nouveau ajouté un bouton << À propos >>,
            déplacé le bouton << accueil >> sur la droite et affiché à sa place le titre du site qui dirige vers l'accueil lors de son activation.\\
            Nous avons également introduit la barre de recherche sur chaque page. On remarque que le bouton << deconnexion >> est bien présent.
        \end{paragraphe}

        \image{0.30}{page_type_2_final}{Disposition 2 - Layout final 2}

	\newpage

	\subsection{Disposition 3 (layout)}

		\begin{paragraphe}
            Ceci est la troisième disposition dont toutes les pages vont hériter dans le cas où un utilisateur administrateur est connecté.
		\end{paragraphe}

        \begin{center}
            \begin{tabular}{l c | c l}
                \textbf{Si} clic sur le bouton << accueil >> & & & \textbf{Si} clic sur le bouton << administration >> \\
                \textbf{Alors} redirection vers \underline{pageAccueil.php}. & & & \textbf{Alors} redirection vers \underline{pageAdminAccueil.php}. \\ \\
                \textbf{Si} clic sur le bouton << compte >> & & & \textbf{Si} clic sur bouton << déconnexion >> \\
                \textbf{Alors} redirection vers \underline{pageCompte.php}. & & & \textbf{Alors} déconnexion de l'utilisateur de la session.
            \end{tabular}
        \end{center}

		\image{0.35}{page_type_3}{Mockup - Layout 3}

        \begin{paragraphe}
            Le layout final obtenu est également très proche de ce que nous avions conceptualisé. Nous avons de nouveau ajouté un bouton << À propos >>,
            déplacé le bouton << accueil >> sur la droite et affiché à sa place le titre du site qui dirige vers l'accueil lors de son activation.\\
            Nous avons également introduit la barre de recherche sur chaque page. On remarque que le bouton << administration >> est bien présent.
        \end{paragraphe}

        \image{0.30}{page_type_3_final}{Disposition 3 - final 3}

    \newpage

\section{Présentation cahier des charges}

\newpage

\section{Définition des pages}

	    	\begin{paragraphe}
	    	    Nous avions initialement prévu les pages suivantes pour composé notre site:\\
        		\begin{itemize}
        			\item \underline{pageAccueil.php} : Page d'accueil de notre site web.
        			\item \underline{pageRésultatRecherche.php} : Page des résultats des recherches entrée dans la barre de recherche par l'utilisateur.
        			\item \underline{pageArtisteGroupe.php} : Page des informations d'un artiste ou d'un groupe ainsi que les albums et/ou singles du groupe ou de l'artiste.
        			\item \underline{pageAlbum.php} : Page d'affichage des donénes d'un morceau de musique.
        		\end{itemize}
                \vspace{1em}
                \begin{itemize}
        			\item \underline{pageConnexion.php} : Permet de se connecter au compte de l'utilisateur.
        			\item \underline{pageInscription.php} : Permet de créer un compte utilisateur.
        			\item \underline{pageCompte.php} : Permet d'afficher les informations de l'utilisateur et offre à l'utilisateur la possibilité de modifier son mot de passe.
        		\end{itemize}
                \vspace{1em}
        		\begin{itemize}
        			\item \underline{adminAccueil.php} : Permet de rediriger l'administrateur sur les différents panneaux de gestion de la base de données.
        			\item \underline{adminGestionArtiste.php} : Permet de gérer les différents artistes, en ajoutant ou en modifiant leurs données. Ainsi que leur attribuer des récompenses.
        			\item \underline{adminGestionGroupe.php} : Permet de gérer les différents groupes, en ajoutant ou en modifiant leurs données. Ainsi que de relier un ou plusieurs artistes à un groupe.
        			\item \underline{adminGestionAlbum.php} : Permet de gérer les différents albums, en ajoutant ou en modifiant leurs données. Ainsi que de relier un ou plusieurs artistes à un album.
        			\item \underline{adminGestionMusique.php} : Permet de gérer les différents morceaux de musique, en ajoutant ou en modifiant leurs données. Ainsi que de relier un ou plusieurs artistes à une musique, relier un ou plusieurs albums à un musique et de relier un ou plusieurs genres à une musique.
        			\item \underline{adminFormGenre.php} : Permet d'ajouter ou de modifier un genre à un morceau de musique ou un single.
        			\item \underline{adminFormRecompense.php} : Permet d'ajuter une récompense à un artiste.
        			\item \underline{adminGestionUtilisateur.php} : Permet à un administrateur préalablement identifié d'ajouter un utilisateur et/ou de changer le statut d'un utilisateur (administrateur ou non).
        		\end{itemize}
        	\end{paragraphe}

        	\begin{paragraphe}
                Cependant au fur et à mesure du développement nous nous sommes aperçu que par souci de modularité, de lisibilité, de factorisation et de simplicité,
                certains fichiers devaient être divisés en plusieurs quand d'autres pouvaient être fusionnés.
                De même certains on était ajoutés pour répondre aux fonctionnalités que nous n'avions pas conceptualisé.
            \end{paragraphe}

            \begin{paragraphe}
                Voici donc, page suivante, une liste exhaustive, tirée de notre documentation phpDoc générée à l'aide de Doxygen, présentant l'ensemble des fichiers .php avec un court descriptif de leur
                fonction rangé dans une arborescence.
            \end{paragraphe}

\newpage

            \image{0.43}{liste_fichier}{Définition des pages - Liste finale des fichiers}

    \clearpage

    \subsection{Pages publiques}

		\subsubsection{Page accueil}

            \begin{paragraphe}
                Nous avions originellement imaginé une page d'accueil contenant une barre de recherche, une liste des nouveaux artistes et des sorties récentes de musiques (Voir Mockup de la page d'accueil).\\
                Nous avons cependant opté pour une conception plus sobre et simple en affichant simplement une grande barre de recherche (Voir page d'accueil finale).
            \end{paragraphe}

			\begin{paragraphe}
                La barre de recherche a une place centrale sur la page d'accueil. Elle permet de saisir des recherches d’artiste, d’album, de groupe ou de morceau de musique.
				En entrant une recherche, les résultats seront affichés directement et différents liens seront disponibles.
                \image{0.50}{accueil_barre_recherche}{Page accueil - Barre recherche}
			\end{paragraphe}

			\begin{paragraphe}
				\textbf{Service PHP :}
            \end{paragraphe}

            \begin{paragraphe}
                Redirection vers la page recherche.php qui traitera la requête soumise dans le formulaire de la barre de recherche.
			\end{paragraphe}

			\begin{paragraphe}
				\textbf{Conception IHM :}
            \end{paragraphe}

            \image{0.30}{page_accueil}{Mockup - Page d'accueil}

            \begin{paragraphe}
                \textbf{Si} clic sur bouton << rechercher >> \\
                \textbf{Alors} redirige vers la page des resultats recherche en passant en paramètre la chaîne présente dans la barre de recherche
            \end{paragraphe}

             \image{0.30}{page_accueil_finale}{Pages publiques - Page d'accueil finale}

        \clearpage

		\subsubsection{Page résultat recherche}

			\begin{paragraphe}
				Cette page fait des requêtes sur la base de données, sur les entités << Artiste >>, << Musique >>, << Groupe>> et << Album >>.
				 en passant en paramètre la chaine de caractère entrée dans la barre de recherche.\\
                En cliquant sur le résultat souhaité, l'utilisateur est redirigé vers la page de l'artiste, du groupe, de l'album ou de la musique
                 recherché où il y trouve des informations détaillées caractérisant l'entité sélectionnée.
            \end{paragraphe}

            \begin{paragraphe}
                \textbf{Service PHP :}
            \end{paragraphe}

            \begin{paragraphe}
                Les services utilisés pour la recherche (Album, artiste, groupe ou musique):
                 \begin{itemize}
                     \item recupererAlbum(nomAlbum) affiche tous les albums correspondant à la chaîne recherchée.
                     \item recupererArtiste(nomArtiste) affiche tous les artistes correspondant à la chaîne recherchée.
                     \item recupererGroupe(nomGroupe) affiche tous les groupes correspondant à la chaîne recherchée.
                     \item recupererMusique(nomMusique) affiche toutes les musiques correspondantes à la chaîne recherchée.
                 \end{itemize}
             \end{paragraphe}

            \begin{paragraphe}
                 Lorsqu'un résultat est sélectionné, pour l'afficher il va falloir utiliser les fonctions :
                \begin{itemize}
                    \item getAlbum(nomAlbum) si un album a été sélectionné.
                    \item getArtiste(nomArtiste) si un artiste a été sélectionné.
                    \item getGroupe(nomGroupe) si un groupe a été sélectionné.
                    \item getMusique(nomMusique) si une musique a été sélectionnée.
                \end{itemize}
            \end{paragraphe}

			\begin{paragraphe}
				\textbf{Conception IHM :}
			\end{paragraphe}

			\begin{paragraphe}
			    Nous avions, lors de la conception initiale, prévu d'afficher les résultats sous forme d'un tableau générique contenant 3 colonnes :
				\begin{itemize}
					\item Nom : le nom des albums, morceaux de musique, groupes ou artistes correspondants à la recherche entrée
					\item Type : type de la recherche : album, morceau, groupe ou artiste
					\item Date : affiche, selon le cas, la date de sortie de la musique ou de l'album, la date de naissance de l'artiste ou la date de formation du groupe
				\end{itemize}
				Chaque résultat de la colonne << Nom >> devait être cliquable, et sur clic, diriger vers la page dédiée au résultat (i.e : la page de l'album, du morceau, du groupe ou de l'artiste correspondant).
			\end{paragraphe}
            
			\image{0.35}{page_resultat_recherche}{Mockup - Page résultats finale}

            \begin{paragraphe}
                \textbf{Si} une recherche est entrée\\
                \textbf{Alors} affichage des résultat correspondants sous forme de liens prévisionnels cliquables\\
                \textbf{Sinon} affiche tout les résultats présent dans la base de donnée.
            \end{paragraphe}

            \begin{paragraphe}
                \textbf{Si} clic sur un résultat\\
                \textbf{Alors} redirection vers la page de la recherche souhaitée en passant en paramètre l'id souhaité.
            \end{paragraphe}

            \begin{paragraphe}
                Nous avons cependant opté, lors du développement, sur affichage en ligne qui nous semblait plus lisible sur les différentes plateformes.\\
                En voici un exemple:
            \end{paragraphe}
            
            \image{0.33}{page_resultat_recherche_finale}{Pages publiques - Page résultat finale}

        \clearpage

		\subsubsection{Page description artiste/groupe}
            \begin{paragraphe}
                Cette page fait des requêtes sur la base de données, sur les entités << Artiste >> ou << Groupe >>.
                 en passant en paramètre le nom transmis par la page resultat de recherche. Elle affiche ensuite ces dernières.
            \end{paragraphe}

            \begin{paragraphe}
                \textbf{Service PHP :}
            \end{paragraphe}

            \begin{paragraphe}
                Pour afficher les différentes métadonnées de l'artiste ou du groupe nous avons besoin des services suivant:
                \begin{itemize}
                        \item getArtiste(nomArtiste) si un artiste a été sélectionné, pour obtenir les informations de l'artiste.
                        \item getGroupe(nomGroupe) si un groupe a été sélectionné, pour obtenir les informations du groupe.
                \end{itemize}
            \end{paragraphe}

			\begin{paragraphe}
				\textbf{Conception IHM :}
			\end{paragraphe}

			\begin{paragraphe}
				Initialement : \\
				On affiche tous les albums et tous les singles de l'artiste. \\
				Avec la fonction getAlbum(artiste), getSingle(Artiste).
			\end{paragraphe}

			\begin{paragraphe}
				\textbf{Si} clic sur le nom d'un album \\
				\textbf{Alors} redirection vers la page pageAlbum.php en passant en paramètre le nom de l'album.
			\end{paragraphe}

			\image{0.35}{page_artiste}{Mockup - Page artiste/groupe}

         \newpage

            \begin{paragraphe}
                La page finale est assez fidèle au modèle que nous avions conceptualisé.
            \end{paragraphe}

            \begin{paragraphe}
                \textbf{Voici un exemple pour un artiste :}
            \end{paragraphe}
            
            \image{0.35}{page_artiste_finale2}{Pages publiques - Page artiste finale}

            \begin{paragraphe}
                \textbf{Voici un autre exemple pour un groupe :}
            \end{paragraphe}
            
            \image{0.35}{page_groupe_finale2}{Pages publiques - Page groupe finale}

        \clearpage

		\subsubsection{Page description album/musique}

			\begin{paragraphe}
                Cette page fait des requêtes sur la base de données, sur les entités << album >> ou << Musique >>.
                 en passant en paramètre le nom transmis par la page resultat de recherche. Elle affiche ensuite ces dernières.
				Si un utilisateur non connecté souhaite rédiger un commenter et donner une note à un album ou à une musique,
				 il faut d'abord qu'il se connecte. Il y a donc dans l'espace commentaire un lien << connectez-vous >> le redirigeant vers la page de connexion.
			\end{paragraphe}

           \begin{paragraphe}
                \textbf{Service PHP :}
            \end{paragraphe}

            \begin{paragraphe}
                Pour afficher les différentes métadonnées de l'artiste ou du groupe nous avons besoin des services suivant:
                \begin{itemize}
                        \item getAlbum(nomAlbum) si un album a été sélectionné, pour obtenir les informations de l'album.
                        \item getMusique(nomMusique) si une musique a été sélectionnée, pour obtenir les informations de la musique.
                \end{itemize}
            \end{paragraphe}

            \begin{paragraphe}
                \textbf{Conception IHM :}
            \end{paragraphe}

			\begin{paragraphe}
				Sur cette page, l'utilisateur pourra mettre un commentaire sur l'album ou la musique dans le cadre consacré au commentaire.
				Dans cet espace, il pourra aussi donner une note et cliquer sur le bouton << Evaluer >> afin que soit pris en compte son commentaire.
			\end{paragraphe}

			\begin{paragraphe}
				\textbf{Si} clic sur sur une des 5 étoiles \\
				\textbf{Alors} la note est définit par le numéro de l'étoile.
			\end{paragraphe}

			\begin{paragraphe}
				\textbf{Si} clic sur le bouton << evaluer >> \\
				\textbf{Alors} envoi commentaire et la note à la base de données et affichage à l'écran du commentaire et de la note visible par tout utilisateur du site.
			\end{paragraphe}

            \image{0.3}{page_album_non_connecter}{Mockup - Page album non connecté}

        \clearpage

            \image{0.3}{page_album_connecter}{Mockup - Page album connecté}

        \clearpage
        
			\begin{paragraphe}
                Les pages finales d'album sont ici aussi très ressemblantes avec notre conception.
			\end{paragraphe}

			\begin{paragraphe}
                \textbf{Voici un exemple pour album avec un utilisateur connecté :}
			\end{paragraphe}
            
            \image{0.33}{page_album_deconnecte2}{Pages publiques - Page album non connecté finale}

			\begin{paragraphe}
                \textbf{Voici le même exemple avec cette fois un utilisateur déconnecté :}
			\end{paragraphe}
            
            \image{0.33}{page_album_connecte2}{Pages publiques - Page album connecté finale}

	\clearpage

	\subsection{Pages utilisateurs}
      \begin{paragraphe}
            Les pages utilisateurs sont relatives à la gestion de compte utilisateur.
        \end{paragraphe}

		\subsubsection{Page de connexion}

			\begin{paragraphe}
				Cette page est accessible à toute personne non connectée à l'application web.
			\end{paragraphe}

			\begin{paragraphe}
				\textbf{Service PHP :}
			\end{paragraphe}

			\begin{paragraphe}
			    Cette page intéragie avec la base de données avec la fonction :
				\begin{itemize}
					\item \emph{connexionUtilisateur(Utilisateur)} : vérifie si un utilisateur correspondant existe dans la base de données.
				\end{itemize}
			\end{paragraphe}

			\begin{paragraphe}
				\textbf{Conception IHM :}
			\end{paragraphe}

			\begin{paragraphe}
				\textbf{Si} l'utilisateur n'est pas connecté \\
				\textbf{Alors} initialement la page est constituée d'un simple formulaire non rempli avec deux champs: nom utilisateur et mot de passe.
			\end{paragraphe}

            \begin{center}
                \begin{tabular}{l c | c l}
                    \textbf{Si} clic sur le bouton << connexion >> & & & \textbf{Si} clic sur lien  << s'inscrire >> \\
                    \textbf{Alors} effectue l'action \emph{connexionutilisateur(Utilisateur).} & & & \textbf{Alors} redirection vers page inscription.
                \end{tabular}
            \end{center}

            \begin{center}
                \begin{tabular}{l}
                    \textbf{Si} l'action précédente a réussi \\
                    \textbf{Alors} l'utilisateur est ensuite redirigé vers la page principale de l'application web. \\
                    \textbf{Sinon} l'utilisateur reste sur la page avec un message d'erreur.
                \end{tabular}
            \end{center}

			\image{0.25}{page_connexion}{Mockup - Page connexion}

        \newpage
        
            \begin{paragraphe}
                Un nouveau lien a été introduit permettant à un utilisateur, ne possédant pas de compte, d'être dirigé vers le formulaire d'inscription
            \end{paragraphe}

            \image{0.60}{page_connexion_finale}{Pages Utilisateurs - Page connexion finale}

    \clearpage

		\subsubsection{Page d'inscription}

			\begin{paragraphe}
				Cette page est accessible à toutes personnes non connectées à l'application web.
			\end{paragraphe}

			\begin{paragraphe}
				\textbf{Service PHP :}
			\end{paragraphe}

			\begin{paragraphe}
			    Cette page intéragie avec la base de données avec la fonction:
				\begin{itemize}
					\item ajouterUtilisateur(Utilisateur) : ajoute un utilisateur dans la base de données.
				\end{itemize}
			\end{paragraphe}

			\begin{paragraphe}
				\textbf{Conception IHM :}
			\end{paragraphe}

			\begin{paragraphe}
				\textbf{Si} l'utilisateur n'est pas connecté \\
				\textbf{Alors} initialement la page est constituée d'un simple formulaire non rempli avec trois champs de saisi: nom utilisateur, mot de passe et vérification mot de passe.
				Il y également un bouton << Inscription >> ainsi qu'un lien redirigeant vers la page connexion intitulé << se connecter >>.
			\end{paragraphe}

			\begin{paragraphe}
				\textbf{Si} clic sur le bouton << inscription >> \\
				\textbf{Alors} effectue l'action \emph{ajouterUtilisateur(Utilisateur)}.
			\end{paragraphe}

           \begin{paragraphe}
                \textbf{Si} clic sur lien << se connecter >> \\
                \textbf{Alors} redirection vers la page \underline{pageConnexion.php}.
            \end{paragraphe}

			\begin{paragraphe}
				\textbf{Si} l'action précédente a réussi \\
				\textbf{Alors} l'utilisateur est ensuite redirigé vers la page \underline{pageConnexion.php} avec le message : (votre compte à bien été enregistré). \\
				\textbf{Sinon} l'utilisateur reste sur la page avec un message d'erreur.
			\end{paragraphe}

            \image{0.3}{page_inscription}{Mockup - Page inscription}

        \newpage
        
            \begin{paragraphe}
                Nous avons de nouveau introduit un lien, permettant à un utilisateur possédant un compte d'être redirigé sur la page \underline{pageConnexion.php}.
            \end{paragraphe}

            \image{0.60}{page_inscription_finale}{Pages Utilisateurs - Page inscription finale}

    \clearpage

		\subsubsection{Page de gestion compte utilisateur}

			\begin{paragraphe}
				Ce formulaire est composé de 2 champs :
				\begin{itemize}
					\item Un champ pour entrer le nouveau mot de passe du compte.
					\item Un champ pour confirmer le nouveau mot de passe du compte (et ainsi éviter les maladresses de saisie).
				\end{itemize}
			\end{paragraphe}

			\begin{paragraphe}
				Après vérification par le biais d'une requête sur la base de données, la page affichera un message indiquant, si toutes les données entrée sont valides, que la modification à bien été effectuée. Ou, si certaines informations sont erronées, la nature de l'erreur et proposera de remplir à nouveau le formulaire avec de bonnes valeurs.
			\end{paragraphe}

			\begin{paragraphe}
				\textbf{Service PHP :}
			\end{paragraphe}

			\begin{paragraphe}
				Les services sont ici utilisés pour la vérification des deux mots de passes saisis. \\
				Une fois cette vérification établie, la base de données est modifiée par la nouvelle valeur correspondant au nom de compte connecté.
			\end{paragraphe}

			\begin{paragraphe}
                Il faut vérifier ici les deux mots de passes saisie. \\
                Une fois cette vérification établie, la base de données est modifiée par la nouvelle valeur correspondant au nom de compte connecté.
				On effectue les vérifications de mot de passe avec la fonction :
				\begin{itemize}
					\item \emph{modofierMotDePasse(MotDePasse)} : Change le mot de passe d'un utilisateur (celui connecté).
				\end{itemize}
			\end{paragraphe}

			\begin{paragraphe}
				\textbf{Conception IHM :}
			\end{paragraphe}

			\begin{paragraphe}
				Initalement, le formulaire est vide, il sera soumis à la page formUtilisateur.php pour la vérification.
			\end{paragraphe}

			\image{0.3}{page_compte}{Page compte}

			\begin{paragraphe}
				Si clic sur le bouton << Changer de mot de passe >> \\
				Alors effectue l'action modifierMotDePasse(MotDePasse).
			\end{paragraphe}

        \newpage
        
            \begin{paragraphe}
               Au cours du développment, l'aspect et la logique de cette page a été conservé. Nous avons cependant ajouté une autre fonctionnalité
               à la gestion de compte. L'utilisateur peut supprimer son compte. Nous avons ajouté une page intermédiaire accessible en cliquant sur le bouton
               << Compte >> présent sur la bandeau d'en tête.
            \end{paragraphe}
               
            \begin{paragraphe}
                Cette page nous indique notre nom d'utilsateur et ainsi que deux boutons, l'un supprime le compte auquel nous somme connecté en transmettant une requête de suppression dans la
                base de données. Le second bouton renvoie l'utilisateur à au formulaire de modification de mot de passe.
            \end{paragraphe}

            \begin{paragraphe}
                \textbf{Voici une illustration de ces deux pages finales :}
            \end{paragraphe}

            \image{0.60}{page_compte_finale}{Pages Utilisateurs - Page gestion utilisateur finale}

            \image{0.60}{page_modif_mdp}{Pages Utilisateurs - Page modification mot de passe finale}

	\clearpage

	\subsection{Pages d'administration}
        
        \begin{paragraphe}
            Dans la version un, du dossier de conception, toutes les pages d'administrations été chargées de formulaires, d'actions et de boutons.
            Lors de la réalisation de l'application nous avons effectué une simplification de toutes les pages, en répartisant et délimitant les périmétres d'action de chaque page.
        \end{paragraphe}
    
        \begin{paragraphe}
            Nous avons donc délimité six catégories pour gérer les différentes entités de la base de donnée :
            \vspace{1em}
            \begin{itemize}
                \item Gestion des utilisateurs
                \item Gestion des artistes
                \item Gestion des groupes
                \item Gestion des albums
                \item Gestion des morceaux de musiques
                \item Gestion des récompenses
            \end{itemize}
        \end{paragraphe}
        
        \begin{paragraphe}
            Chaques pages des six catégories est rassemblé dans un dossier portant le nom de l'entité qu'il gère. \\
            De plus une page gestion, une page formulaire et une page action pour chaque catégorie permettent de séparer le code pour une meilleur lisibilité.
        \end{paragraphe}

        \begin{paragraphe}
            \textbf{Les pages gestions} permettent de gérer les valeurs stocké dans la base de donnée, en les modifiant ou en les supprimant.
        \end{paragraphe}
        
        \begin{paragraphe}
            \textbf{Les pages formulaires} permettent d'ajouter ou de modifier les données stocker des entités.
        \end{paragraphe}
        
        \begin{paragraphe}
            \textbf{Les pages actions} permettent quand à elles de d'efectuer des vérifications, des validations et d'exécuter les actions souhaité.
        \end{paragraphe}

		\subsubsection{Page d'administration accueil}

			\begin{paragraphe}
				\textbf{Service PHP :}
			\end{paragraphe}

			\begin{paragraphe}
				Dans cette page, il n'y a pas de service PHP rendu. Il y a seulement des redirections vers d'autres pages.
			\end{paragraphe}

			\begin{paragraphe}
                On a supprimer les deux boutons d'ajout du genre et de récompense et seulement ajouter un bouton pour la gestion des récompenses.
			\end{paragraphe}
            
			\begin{paragraphe}
				\textbf{Conception IHM :}
			\end{paragraphe}

			\begin{paragraphe}
				Cette page hérite de la disposition n°3. \par
				Initialement cette page contient sept boutons :
				\begin{itemize}
					\item Gestion artistes.
					\item Gestion groupes.
					\item Gestion albums.
					\item Gestion musiques.
					\item Gestion utilisateurs.
                    \item Gestion récompense.
				\end{itemize}
			\end{paragraphe}

            \begin{center}
                \begin{tabular}{l | l}
                    \textbf{Si} clic sur le bouton << Gestion artistes >> & \textbf{Si} clic sur le bouton << Gestion groupes >> \\
                    \textbf{Alors} redirection vers adminGestionArtiste.php. & \textbf{Alors} redirection vers adminGestionGroupe.php. \\ \\
                    
                    \textbf{Si} clic sur le bouton << Gestion albums >> & \textbf{Si} clic sur le bouton << Gestion musiques >> \\
                    \textbf{Alors} redirection vers admingGestionAlbum.php. & \textbf{Alors} redirection vers adminGestionMusique.php. \\ \\
                    
                    \textbf{Si} clic sur le bouton << Gestion utilisateur >> & \textbf{Si} clic sur le bouton << Gestion récompense >> \\
                    \textbf{Alors} redirection vers adminGestionutilisateur.php. & \textbf{Alors} redirection vers adminGestionRecompense.php.
                \end{tabular}
            \end{center}

        \clearpage

            \begin{paragraphe}
                \textbf{Voici le Mockup prévu :}
            \end{paragraphe}

			\image{0.35}{page_admin_index}{Mockup - Page d'administration accueil}
            
            \begin{paragraphe}
                \textbf{Voici le nouveau panneau d'administration :}
            \end{paragraphe}
            
			\image{0.35}{page_admin_index_final}{Page administration - Accueil final}
            
    \clearpage

		\subsubsection{Page gestion artiste}

			\begin{paragraphe}
				Cette page est accesible seulement à un utilisateur disposant du statut d'administrateur.
			\end{paragraphe}

			\begin{paragraphe}
				\textbf{Service PHP :}
			\end{paragraphe}

			\begin{paragraphe}
				Voici la liste des fonctios utilisées pour cette page PHP interagissant avec la base de données :
			\end{paragraphe}

			\begin{paragraphe}
				\begin{itemize}
					\item getListeArtiste() : récupère dans la base de données, l'ensemble des artistes inscrit.
					\item getArtiste(artiste) : récupère l'entrée dans la base de données d'un artiste en particulier.
				\end{itemize}
			\end{paragraphe}

			\begin{paragraphe}
				\begin{itemize}
					\item ajouterNouveauArtiste(artiste) : permet d'enregistrer un nouvel << artiste >> dans la base de donnée.
					\item modifierArtiste(artiste) : permet de mettre à jour une entrée << artiste >> déjà existante dans la base de données.
					\item suppressionArtiste(artiste) : permet de supprimer un artiste de la base de données.
					\item ajoutRecmpense(artiste, recompense) : permets d'ajouter une récompense à un artiste.
				\end{itemize}
			\end{paragraphe}

			\begin{paragraphe}
				\textbf{Conception IHM :}
			\end{paragraphe}

			\begin{paragraphe}
				Cette page hérite de la disposition n°3. \par
				Initialement les formulaires de cette page sont vides.\par
				Le tableau est rempli à l'aide de l'action getListeArtiste().
			\end{paragraphe}

			\image{0.35}{page_admin_artiste}{Mockup - Page gestion artistes}

			\begin{paragraphe}
				Il y a deux formulaires. Un premier qui permet de modifier les données d'un artiste, et le second pour ajouter des récompenses à l'utilisateur sélectionné.
			\end{paragraphe}

			\begin{paragraphe}
				Premier formulaire :
			\end{paragraphe}

			\begin{paragraphe}
				Si clic sur le tableau d'artiste, le bouton << selectionner >> \\
				Alors effectuer l'action getArtiste(artiste).
			\end{paragraphe}

			\begin{paragraphe}
				Si clic sur le tableau d'artiste, le bouton << supprimer >> \\
				Alors effectue l'action supprimerArtiste(artiste).
			\end{paragraphe}

			\begin{paragraphe}
				Si clic sur le bouton << Ajouter >> du formualire \\
				Alors effectue l'action ajouterNouveauArtiste(artiste).
			\end{paragraphe}

			\begin{paragraphe}
				Si clic sur le bouton modifier du formulaire \\
				Alors effectuer l'action modifierArtiste(artiste).
			\end{paragraphe}

			\begin{paragraphe}
				Le second formulaire :
			\end{paragraphe}

			\begin{paragraphe}
				Si aucun artiste n'est sélectionné \\
				Alors ce formulaire n'existe pas \\
				Sinon
			\end{paragraphe}

			\begin{paragraphe}
				Si clic sur la << comboBox >> récompense \\
				Alors une liste déroulante permet de sélectionner les récompenses existantes.
			\end{paragraphe}

			\begin{paragraphe}
				Si clic sur le bouton << ajouter récompense >> \\
				Alors effectuer l'action ajoutRecompense(artiste, recompense).
			\end{paragraphe}

			\begin{paragraphe}
				Si clic sur le bouton << ajouter ue nouvelle récompense >> \\
				Alors on redirige l'administrateur sur la page adminFormRecompense.php
			\end{paragraphe}

			\begin{paragraphe}
			\end{paragraphe}

		\subsubsection{Page gestion groupe}

			\begin{paragraphe}
				Cette page est accessible seulement à un utilisateur disposant du statu administrateur.
			\end{paragraphe}

			\begin{paragraphe}
				\textbf{Service PHP :}
			\end{paragraphe}

			\begin{paragraphe}
				Voici la liste des fonctions utilisées pour cette page PHP interagissant avec la base de données :
			\end{paragraphe}

			\begin{paragraphe}
				\begin{itemize}
					\item getListGroupes() : récupère dans la base de données, l'ensemble des groupes inscrit.
					\item getGroupe(groupe) : recupère l'entrée dans la base de données d'un groupe en particulier. Ainsi que les artistes associés.
				\end{itemize}
			\end{paragraphe}

			\begin{paragraphe}
				\textbf{Conception IHM :}
			\end{paragraphe}

			\begin{paragraphe}
				Cette page hérite de la disposition n°3.\par
				Initalement les formulaires de cette page sont vides.
				Le tableau est rempli à l'aide de l'action getListeGroupes().
			\end{paragraphe}

			\image{0.3}{page_admin_groupe}{Mockup - Page gestion groupe d'artiste}

			\begin{paragraphe}
				Il y a deux formulaires. Un premier qui permet de modifier les données d'un groupe, et le second pour ajouter des artistes au groupe selectionné.
			\end{paragraphe}

			\begin{paragraphe}
				Premier formulaire :
			\end{paragraphe}

			\begin{paragraphe}
				Si clic sur le tableau de groupe, le bouton << sélectionner >> \\
				Alors effectuer l'action getGroupe(groupe).
			\end{paragraphe}

			\begin{paragraphe}
				Si clic sur le tableau de groupe, le bouton << supprimer >> \\
				Alors effectuer l'action suppressionGroupe(groupe).
			\end{paragraphe}

			\begin{paragraphe}
				Si clic sur le bouton << Ajouter >> du formulaire \\
				Alors effectuer l'action ajouterNouveauGroupe(groupe).
			\end{paragraphe}

			\begin{paragraphe}
				Si clic le bouton << Modifier >> du formulaire \\
				Alors effectuer l'action modifierGroupe(groupe).
			\end{paragraphe}

			\begin{paragraphe}
				Le second formulaire :
			\end{paragraphe}

			\begin{paragraphe}
				Si aucun groupe n'est sélectionné \\
				Alors ce formulaire n'existe pas. \\
				Sinon
			\end{paragraphe}

			\begin{paragraphe}
				Si clic sur la << comboBox >> artites \\
				Alors une liste déroulante permet de sélectionner les artistes existants.
			\end{paragraphe}

			\begin{paragraphe}
				Si clic sur le bouton << ajouter artiste >> \\
				Alors effectuer l'action ajoutArtisteGroupe(groupe, artiste).
			\end{paragraphe}

			\begin{paragraphe}
				Si clic sur le bouton << ajouter un nouvel artiste >>
				Alors on redirige l'administrateur sur la page adminGestionArtiste.php
			\end{paragraphe}

		\subsubsection{Page gestion album}

			\begin{paragraphe}
				Cette page est accessible seulement à un utilisateur disposant du statut administrateur.
			\end{paragraphe}

			\begin{paragraphe}
				\textbf{Service PHP :}
			\end{paragraphe}

			\begin{paragraphe}
				Voici la liste des fonctions utilisées pour cette page PHP interagissant avec la base de données :
			\end{paragraphe}

			\begin{paragraphe}
				\begin{itemize}
					\item getListeAlbum() : récupère dans la base de données l'ensemble des albums inscrit.
					\item getAlbum(groupe) : récupère l'entrée dans la base de données d'un album en particulier. Ainsi que les artiste associés.
					\item ajouterNouvelAlbum(album) : permet d'enregistrer un nouvel << album >> dans la base de données.
					\item modifierAlbum(album) : permet de mettre à jour une entrée << album >> déjà existante dans la base de données.
					\item suppressionAlbum(album) : permet de supprimer un album ainsi que l'association des artiste a l'album.
					\item ajoutArtisteAlbum(album, artiste) : permet d'ajouter un artiste existant à un album.
				\end{itemize}
			\end{paragraphe}

			\begin{paragraphe}
				\textbf{Conception IHM :}
			\end{paragraphe}

			\begin{paragraphe}
				Cette page hérite de la disposition n°3. \\
				Initialement les formulaires de cette page sont vides. \\
				Le tableau est rempli à l'aide de l'action getListeAlbum().
			\end{paragraphe}

			\image{0.3}{page_admin_album}{Mockup - Page gestion album}

			\begin{paragraphe}
				Il y a deux formulaires. Un premier qui permet de modifier les données d'un album, et le second pour ajouter des artistes a l'album selectionné.
			\end{paragraphe}

			\begin{paragraphe}
				Premier formulaire :
			\end{paragraphe}

			\begin{paragraphe}
				Si clic sur le tableau d'album, le bouton << sélectionner >> \\
				Alors effectuer l'action getAlbum(album).
			\end{paragraphe}

			\begin{paragraphe}
				Si clic sur la tableau d'album, le bouton << supprimer >> \\
				Alors effectuer l'action suppressionAlbum(album).
			\end{paragraphe}

			\begin{paragraphe}
				Si clic sur le bouton << Ajouter >> du formulaire \\
				Alors effectuer l'action ajouterNouvelAlbum(album).
			\end{paragraphe}

			\begin{paragraphe}
				Si clic sur le bouton << Modifier >> du formulaire \\
				Alors effectuer l'action modifierAlbum(album).
			\end{paragraphe}

			\begin{paragraphe}
				Le second formulaire :
			\end{paragraphe}

			\begin{paragraphe}
				Si aucun album n'est sélectionné \\
				Alors ce formulaire n'existe pas. \\
				Sinon
			\end{paragraphe}

			\begin{paragraphe}
				Si clic sur la << comboBox >> artistes \\
				Alors une liste déroulante permet de sélectionner les artistes existants.
			\end{paragraphe}

			\begin{paragraphe}
				Si clic sur le bouton << ajouter artiste >> \\
				Alors effectuer l'action ajoutArtisteAlbum(album, artiste).
			\end{paragraphe}

		\subsubsection{Page gestion musique}

			\begin{paragraphe}
				Cette page est accessible seulement à un utilisateur disposant du statut administrateur.
			\end{paragraphe}

			\begin{paragraphe}
				\textbf{Service PHP :}
			\end{paragraphe}

			\begin{paragraphe}
				Voici la liste des fonctions utilisées pour cette page PHP interagissant avec la base de données :
			\end{paragraphe}

			\begin{paragraphe}
				\begin{itemize}
					\item getListeMusique() : récupère dans la base de données, l'ensemble des musiques inscrit.
					\item getMusique(musique) : récupère l'entrée dans la base de données d'une musique en particulier. Ainsi qu les artistes associés, les albums associés et les genres associés.
					\item ajouterNouvelleMusique(musique) : permet d'enregistrer une nouvelle << musique >> dans la base de données.
					\item modifierMusique(musique) : permet de mettre à jour une entrée << musique >> déjà existante dans la base de données.
					\item suppressionMusique(musique) : permet de supprimer une musique ainsi que l'association des artistes à la musique, l'album à la musique et l'association entre les genre et les musiques.
					\item ajoutArtisteMusique(musique, artiste) : permet d'ajouter un artiste existant, à une musique.
					\item ajoutAlbumMusique(musique, album) : permet d'ajouter un album existant, à une musique.
					\item ajoutAGenreMusique(musique, genre) : permet d'ajouter un genre existant, à une musique.
				\end{itemize}
			\end{paragraphe}

			\begin{paragraphe}
				\textbf{Conception IHM :}
			\end{paragraphe}

			\begin{paragraphe}
				Cette page hérite de la disposition n°3. \par
				Initialement les formulaires de cette page sont vides. \par
				Le tableau est rempli à l'aide de l'action getListeAlbum().
			\end{paragraphe}

			\image{0.35}{page_admin_musique}{Mockup - Page gestion musique}

			\begin{paragraphe}
				Il y a quatre formulaires. Un premier qui permet de modifier les données d'un morceau de musique, le second pour ajouter des artistes à une msuique sélectionnée, un troisième pour ajouter la musique à un ou plusieurs albums. Le quatrième formulaire permet d'associé un ou plusieurs genres au morceau de musique.
			\end{paragraphe}

			\begin{paragraphe}
				Premier formualire :
			\end{paragraphe}

			\begin{paragraphe}
				Si clic sur le tableau de musique, le bouton << sélecitionner >> \\
				Alors effectuer l'action getMusique(musique).
			\end{paragraphe}

			\begin{paragraphe}
				Si clic sur le tableau de musique, le bouton << supprimer >> \\
				Alors effectuer l'action suppressionMusique(musique).
			\end{paragraphe}

			\begin{paragraphe}
				Si clic sur le bouton << Ajouter >> du formulaire \\
				Alors effectuer l'action ajouterNouvelleMusique(musique).
			\end{paragraphe}

			\begin{paragraphe}
				Si clic sur le bouton << Modifier >> du formulaire \\
				Alors effectuer l'action modifierMusique(musique).
			\end{paragraphe}

			\begin{paragraphe}
				Le second formulaire :
			\end{paragraphe}

			\begin{paragraphe}
				Si aucune musique n'est sélectionnée \\
				Alors ce formulaire n'existe pas. \\
				Sinon
			\end{paragraphe}

			\begin{paragraphe}
				Si clic sur la << comboBox >> artiste \\
				Alors une liste déroulante permet de sélectionner les artistes existants.
			\end{paragraphe}

			\begin{paragraphe}
				Si clic sur le bouton << ajouter artiste >> \\
				Alors effectuer l'action ajoutArtisteMusique(musique, artiste).
			\end{paragraphe}

			\begin{paragraphe}
				Si clic sur le bouton << ajouter un nouvel artiste >> \\
				Alors redirige l'utilisateur vers la page adminGestionArtiste.php.
			\end{paragraphe}

			\begin{paragraphe}
				Le troisième formulaire :
			\end{paragraphe}

			\begin{paragraphe}
				Si aucune musique n'est sélectionnée \\
				Alors ce formulaire n'existe pas. \\
				Sinon
			\end{paragraphe}

			\begin{paragraphe}
				Si clic sur la << comboBox >> album \\
				Alors effectuer l'action ajoutAlbumMusique(musique, album).
			\end{paragraphe}

			\begin{paragraphe}
				Si clic sur le bouton << ajouter album >> \\
				Alors effectuer l'acion ajoutAlbumMusique(musique, album).
			\end{paragraphe}

			\begin{paragraphe}
				Si clic sur le bouton << ajouter un nouvel album >> \\
				Alors redirige l'utilisateur vers la page adminGestionAlbum.php
			\end{paragraphe}

			\begin{paragraphe}
				Le quatrième formulaire :
			\end{paragraphe}

			\begin{paragraphe}
				Si aucune musique n'est sélectionnée \\
				Alors ce formulaire n'existe pas. \\
				Sinon
			\end{paragraphe}

			\begin{paragraphe}
				Si clic sur la << comboBox >> genre \\
				Alors une liste déroulante permet de sélectionner les genres existants.
			\end{paragraphe}

			\begin{paragraphe}
				Si clic sur le bouton << ajouter genre >> \\
				Alors effetuer l'action ajoutGenreMusique(musique, genre).
			\end{paragraphe}

			\begin{paragraphe}
				Si clic sur le bouton << créer un nouveau genre >> \\
				Alors redirige l'utilisateur vers la page adminFormAlbum.php
			\end{paragraphe}

	\clearpage

		\subsubsection{Page formulaire genre}

			\begin{paragraphe}
				Cette page est accessible seulement à un utilisateur disposant du statut administrateur.
			\end{paragraphe}

			\begin{paragraphe}
				\textbf{Service PHP :}
			\end{paragraphe}

			\begin{paragraphe}
				Voici la liste des fonctions utilisées pour cette page PHP interagissant avec la base de données :
				\begin{itemize}
					\item getGenre(musique, genre) : récupère le genre d'une musique en particulier.
					\item ajouterGenre(musique, genre) : ajout un genre à un morceau de musique.
					\item modifierGenre(musique, genre) : modifier un genre d'un morceau de musique.
				\end{itemize}
			\end{paragraphe}

			\begin{paragraphe}
				\textbf{Conception IHM :}
			\end{paragraphe}

			\begin{paragraphe}
				Soit :
				\begin{itemize}
					\item La page reçoit en paramètre un identifiant d'un morceau << morceau de musique >> (seulement) et auquel cas il s'agit de créer un nouveau << genre >> pour une musique.
					\item La page reçoit en paramètre un identifiant d'un << morceau de musique >> ainsi qu'un identifiant d'un << genre >> auquel cas il s'agit d'une modification de << genre >>.
					\item la page reçoit aucun paramètre. Il s'agit d'une erreur, on retourne à la page précédente.
				\end{itemize}
			\end{paragraphe}

			\image{0.3}{page_admin_genre}{Mockup - Page formulaire genre}

			\begin{paragraphe}
				Si on est dans le premier cas \\
				Alors on laisse le formulaire vide.
			\end{paragraphe}

			\begin{paragraphe}
				Sinon Si on est dans le second cas \\
				Alors on récupère les données du formulaire à l'aide des identifiants (musique et genre) récupérés par la page et on effectue l'action getGenre(musique, genre).
			\end{paragraphe}

			\begin{paragraphe}
				Si clic sur le bouton << Ajouter >> \\
				Alors effectuer l'action ajouterGenre(musique, genre), redirige ensuite l'utilisateur vers la page de musique précédente.
			\end{paragraphe}

			\begin{paragraphe}
				Si clic sur le bouton << Modifier >> \\
				Alors effectuer l'action modifierGenre(musique, genre), redirige ensuite l'utilisateur vers la page de musique précédente.
			\end{paragraphe}

	\clearpage

		\subsubsection{Page formulaire récompense}

			\begin{paragraphe}
				Cette page est accessible seulement à un utilisateur disposant du statut administrateur.
			\end{paragraphe}

			\begin{paragraphe}
				\textbf{Service PHP :}
			\end{paragraphe}

			\begin{paragraphe}
				Voici la liste des fonctions utilisées pour cette page PHP interagissant avec la base de données :
				\begin{itemize}
					\item getRecompense(artiste, recompense) : récupère toutes les informations d'une récompense paricuilère d'un artiste.
					\item ajouterRecompense(artiste, recompense) : ajoute une récompense à un artiste en particulier.
					\item modifierRecompense(artiste, recompense) : modifier une récompense à un artiste en particulier.
				\end{itemize}
			\end{paragraphe}

			\begin{paragraphe}
				\textbf{Conception IHM :}
			\end{paragraphe}

			\begin{paragraphe}
				Soit :
				\begin{itemize}
					\item La page reçoit en paramètre un identifiant d'un << artiste >> (seulement), auquel cas il s'agit de créer une nouvelle << récompense >> pour l'artiste.
					\item La page reçoit en paramètre un identifiant d'un << artiste >> ainsi qu'un identifiant d'une << récompense >>, auquel cas il s'agit d'une modification d'une << récompense >>.
					\item La page ne reçoit aucun paramètre. Il s'agit d'une erreur, on retourne à la page précédente.
				\end{itemize}
			\end{paragraphe}

			\image{0.3}{page_admin_recompense}{Mockup - Page formulaire récompense}

			\begin{paragraphe}
				Si on est dans le premier cas \\
				Alors formulaire reste vide.
			\end{paragraphe}

			\begin{paragraphe}
				Si on est dans le second cas \\
				Alors effectuer l'action getRecompense(artiste, recompense) à l'aide des deux identifiants passé en paramètre.
			\end{paragraphe}

			\begin{paragraphe}
				Si clic le bouton << Ajouter >> \\
				Alors effectuer l'action ajouterRecompense(artiste, recompense), redirige ensuite l'utilisateur vers la page de l'artiste précédent.
			\end{paragraphe}

			\begin{paragraphe}
				Si clic sur le bouton << Modifier >> \\
				Alors effectuer l'action modifierRecompense(artiste, musique), redirige ensuite l'utilisateur vers la page de l'artiste précédent.
			\end{paragraphe}

		\subsubsection{Page gestion compte utilisateur}

			\begin{paragraphe}
				Cette page est accessible seulement à un utilisateur disposant du statut administrateur.
			\end{paragraphe}

			\begin{paragraphe}
				\textbf{Service PHP :}
			\end{paragraphe}

			\begin{paragraphe}
				Voici la liste des fonctions utilisées pour cette page PHP interagissant avec la base de données :
				\begin{itemize}
					\item ajouterUtilisateur() : ajoute un utilisateur à la base de données.
					\item getListeUtilisateur() : récupère dans la base de données, l'ensemble des utilisateur inscrit.
					\item getStatutUtilisateur() : récupère le statut d'un utilisateur.
					\item changerStatutUtilisateur() : Modifie le statut d'un utilisateur.
					\item afficherListeUtilisateur() : Affiche sous forme d'un tableau la liste des utilisateurs.
				\end{itemize}
			\end{paragraphe}

			\begin{paragraphe}
				\textbf{Conception IHM :}
			\end{paragraphe}

			\begin{paragraphe}
				La page de gestion des comptes utilisateurs récupère la liste de tous les utilisateurs déjà enregistrés dans la base de données, en affichant le nm de l'utilisateur et son statut (administrateur ou utilisateur normal).
			\end{paragraphe}

			\begin{paragraphe}
				Il y aura un bouton permettant de changer le statut de l'utilisateur << Administrateur >> vers << Utilisateur normal >> et inversement << Utilisateur normal >> vers << Administrateur >>. Le tableau << Liste des utilisateurs >> est rempli par la fonction getListeUtilisateur().
			\end{paragraphe}

			\image{0.35}{page_admin_compte}{Mockup - Page gestion compte utilisateur}

			\begin{paragraphe}
				La page a été découpée en deux :
			\end{paragraphe}

			\begin{paragraphe}
				Si clic sur le bouton << Ajouter >> \\
				Alors effectuer l'action ajouterUtilisateur() si les paramètres entrée dans le formulaire sont valides, recherge ensuite la page après avoir indiqué que l'utilisateur a bien été ajouté. \\
				Sinon signal que les paramètres entrés sont invalides.
			\end{paragraphe}

			\begin{paragraphe}
				Si clic sur le bouton << Changer Statut >> \\
				Alors effectuer l'action changerStatutUtilisateur() si les paramètres entrés dans le formulaire sont valide, recherge ensuite la page après avoir indiqué que le statut a bien été changé.
			\end{paragraphe}
